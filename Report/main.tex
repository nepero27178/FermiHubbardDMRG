\documentclass[a4paper]{article}

\input{misc/packages}
\newcommand{\beq}{\begin{equation}}
\newcommand{\eeq}{\end{equation}}

\newcommand{\red}[1]{\color{red}{#1}}

\newcommand{\mref}{{\color{red}0.0}} % missing ref
\newcommand{\todo}{{\color{tabred}[To be continued\dots]}}

\definecolor{tabred}{RGB}{214, 39, 40}
\definecolor{tabblue}{HTML}{1f77b4}

%\usepackage[usenames,dvipsnames]{color} % more flexible names for syntax highlighting colors

\lstset{
    basicstyle=\ttfamily, 
    numbers=left, 
    numberstyle=\small\ttfamily\color{ForestGreen},
    stepnumber=1,              
    numbersep=10pt, 
    numberfirstline=true, 
    numberblanklines=true, 
    tabsize=4,
    lineskip=0pt,
    aboveskip=10pt,
    belowskip=10pt,
    extendedchars=true,
    breaklines=true,
    %backgroundcolor=\color{ForestGreen!10},
    keywordstyle=\color{blue}\bfseries,
    identifierstyle=, % using emph or index keywords
    commentstyle=\color{ForestGreen},
    stringstyle=\color{Maroon},
    showstringspaces=false,
    showtabs=false,
    upquote=false,
    texcl=true % interpet comments as LaTeX
}

\lstdefinelanguage{julia}
{
  keywordsprefix=\@,
  morekeywords={
    exit,whos,edit,load,is,isa,isequal,typeof,tuple,ntuple,uid,hash,finalizer,convert,promote,
    subtype,typemin,typemax,realmin,realmax,sizeof,eps,promote_type,method_exists,applicable,
    invoke,dlopen,dlsym,system,error,throw,assert,new,Inf,Nan,pi,im,begin,while,for,in,return,
    break,continue,macro,quote,let,if,elseif,else,try,catch,end,bitstype,ccall,do,using,module,
    import,export,importall,baremodule,immutable,local,global,const,Bool,Int,Int8,Int16,Int32,
    Int64,Uint,Uint8,Uint16,Uint32,Uint64,Float32,Float64,Complex64,Complex128,Any,Nothing,None,
    function,type,typealias,abstract
  },
  sensitive=true,
  morecomment=[l]{\#},
  morestring=[b]',
  morestring=[b]" 
}


\addbibresource{misc/bibliography.bib}

\begin{document}

\begin{center}
    {\bfseries {\Large Finite-size DMRG characterization of the}} \\[0.8em]
    {\bfseries {\Large 1D Fermi-Hubbard model phase diagram}}\\[0.8em]
    {\bfseries {\Large under bosonization framework}} \\[1em]
    \large University of Pisa, a.y.~2024-2025 \\[0.8em]
    Alessandro Gori\footnote{\href{mailto:a.gori23@studenti.unipi.it}{a.gori23@studenti.unipi.it} / \href{https://github.com/nepero27178}{nepero27178@github.com}} \\[0.8em]
    \scriptsize Final project for the Master's Degree course ``Quantum Liquids''
    
\end{center}

\renewcommand*{\thefootnote}{\arabic{footnote}}
\setcounter{footnote}{0}

\begin{abstract}
    The one-dimensional Fermi-Hubbard model at zero temperature is studied, employing finite-size DMRG algorithm to investigate some of its ground state properties. The model contains a hopping term between neighbouring sites, a finite on-site interaction energy, and a chemical potential. In order to investigate the zero-temperature ground-state properties of the model, finite-size DMRG was used. 
    
    The entire project heavily relies upon the precedent project carried out by the author together with Marco Pompili, where the 1D Bose-Hubbard model was studied using finite-size DMRG. You may find \href{https://github.com/mrc-pop/BoseHubbardDMRG}{at this link} our previous work.\\
    \begin{center}
        {\footnotesize All of the code can be found at open-access in \href{https://github.com/nepero27178/FermiHubbardDMRG}{this repository}:\\ \url{https://github.com/nepero27178/FermiHubbardDMRG}}
    \end{center}
\end{abstract}
\tableofcontents

\clearpage

\section{Theoretical introduction to bosonization}

This first, vast section is devoted to the introduction of an extremely powerful technique in one-dimensional fermionic problems, namely, bosonization. It is widely based on the comprehensive work of \citeauthor{giamarchi2003quantum}, \cite{giamarchi2003quantum}. I won't get deep in the calculations neither in refined points about the method, being this last arbitrarily vast. The first part of this section deals with the spinless (a.k.a. polarized) case; the second part introduces the spin degree of freedom.

\subsection{Bosonization in a nutshell for spinless fermions}

\begin{figure}
	\centering
	\begin{tikzpicture}
	\begin{axis}[
			axis lines=center,
			xlabel={$k$},
			ylabel={Energy},
			xlabel style={below},
			ylabel style={left},
			xtick=\empty,
			ytick=\empty,
			xmin=-3.5, xmax=3.5,
			ymin=-1.0, ymax=2.5
		]
		\addplot[domain=-3:3, smooth, color=tabblue]
			{1-cos(deg(x))} node[below] {$\epsilon_k$};
		
		\addplot[domain=-3:3, color=tabred]
			{1} node[above] {$\epsilon_F$};
			
		\addplot[domain=0.2:-2.8, color=tabgreen]
			{-0.57-x} node[right] {$\epsilon_k^{(\mathrm L)}$};
		
		\addplot[domain=-0.2:2.8, color=tabgreen]
			{-0.57+x} node[left] {$\epsilon_k^{(\mathrm R)}$};
	\end{axis}
\end{tikzpicture}
	\caption{Sketch of the fermionic band $\epsilon_k$, the Fermi level $\epsilon_F$ and the two linear bands $\epsilon_k^{(\mathrm{L}/\mathrm{R})}$ used to approximate the original bands around the Fermi surface. The double-linear model is often referred to as the Tomonaga-Luttinger model.}
	\label{fig:bands}
\end{figure}

The key idea is simple: start from a conventional fermionic-metallic hamiltonian,
\[
	\hat H = \hat H_0 + \hat V = \sum_k \xi_k \hat c_k^\dagger \hat c_k + \hat V
\]
(we will leave the interaction unspecified, for a moment) where $\xi_k = \epsilon_k - \epsilon_F$ and we are using spinless fermions, for normal bands in ordinary fillings.
Consider Fig.~\ref{fig:bands}: the approximation in the above equation is exactly given by making the following assumption: since at low temperature (which, in metals, is a very broad definition) all the relevant Physics takes place at $\xi \sim 0$, and both the deep-down/far-away single-particle states do not contribute either due to Pauli pressure or state depletion, we can as well study the model:
\[
	\epsilon_k \to \left\{\epsilon_k^{(\mathrm L)}, \epsilon_k^{(\mathrm R)} \right\}
\]
Let $s$ be the side index, $s \in \left\{\mathrm L, \mathrm R\right\}$, with
\[
\sgn(s) = \begin{cases}
	+1 \quad \text{if} \quad s = \mathrm R \\
	-1 \quad \text{if} \quad s = \mathrm L
\end{cases}
\]
Then we may approximate around the Fermi surface (in one dimension degenerated in two points)
\[
	\hat H_0 \simeq \hat K_0 \equiv \sum_{s \in \left\{\mathrm L, \mathrm R\right\}} \sum_k \hbar v_F \left( \sgn(s) k -  k_F \right) \left[ \hat c_k^{(s)} \right]^\dagger \hat c_k^{(s)}
\]
being $k_F \equiv \sqrt{2m\epsilon_F/\hbar}$ the Fermi wavevector and $v_F \equiv \hbar k_F / m$. $\hat K_0$ is the Tomonaga-Luttinger model. In the following, we will set $\hbar=1$. Now, consider the side-wise density operators,
\[
	\hat \rho_q^{(s)} \equiv \sum_k  \left[ \hat c_{k+q}^{(s)} \right]^\dagger \hat c_k^{(s)}
\]
Let us use a slightly different, somewhat lighter notation:
\[
	\hat \rho_q^{(s)} \leftrightarrow \hat \rho_s (q)
\]
From now on, we will proceed only highlighting the important result in the bosonization procedure, since all the detailed derivation is included in \cite{giamarchi2003quantum}.

\subsubsection*{Boson operators}

The pivotal result in the bosonization technique is the following:
\beq\label{eq:density-commutator}
	\comm{\hat \rho_s (q)}{\hat \rho_{s'} (-q')} = - \delta_{ss'} \delta_{qq'} \sgn(s) \frac{qL}{2\pi}
\eeq
where $L$ is the one-dimensional system length.
To get to this point, the very key passage is to employ the identity
\[
	\hat A \hat B = \colon \hat A \hat B \colon + \mel{\Omega}{\hat A \hat B}{\Omega}
\]
being $\hat A$, $\hat B$ two operators made of constructions/destructions, $\ket{\Omega}$ the generic many-body vacuum and $\colon \cdots \colon$ the normal ordering operation. Eq.~\eqref{eq:density-commutator} only holds if we use this trick, which is, if we make a smart use of the infinite particle populations for the linearized model.

Now, Eq.~\eqref{eq:density-commutator} looks ``bosonic''. Notice that the left-side density operator vanishes identically for any $q>0$ on the ground-state Fermi sea $\ket{\Omega}$. This is because it would require to destroy a fermion at any given $k$ and creating one at $k+q$ -- but the monotonicity of the linear left band prevents from doing so, because states on the left get deeper and deeper and thus are already occupied. In formulas
\[
\begin{aligned}
	\hat \rho_\mathrm{L} (q>0) \ket{\Omega} &= 0 \\
	\hat \rho_\mathrm{R} (q<0) \ket{\Omega} &= 0
\end{aligned}
\]
Then, we can define boson operator with finite particle numbers as
\[
\begin{aligned}
	\hat b_q^\dagger &\equiv \sqrt{\frac{2\pi}{\abs{q}L}} \sum_{s \in \left\{\mathrm L, \mathrm R\right\}} \theta \left(\sgn(s) q\right) \hat \rho_s^\dagger (q) \\
	\hat b_q &\equiv \sqrt{\frac{2\pi}{\abs{q}L}} \sum_{s \in \left\{\mathrm L, \mathrm R\right\}} \theta \left(\sgn(s) q\right) \hat \rho_s^\dagger (-q)
\end{aligned}
\]
which of course satisfy
\[
	\comm{\hat b_q}{\hat b_{q'}^\dagger} = \delta_{qq'}
\]

With a little patience, it can be shown that, taking $q \neq 0$,
\beq\label{eq:boson-hamiltonian-commutator}
	\comm{\hat b_q}{\hat K_0} = v_F \abs{q} \hat b_q
\eeq
Assuming the (operatorial) basis generated by the bosonic operators to be complete, then this equation completely defines $\hat K_0$. It must hold:
\[
	\hat K_0 = \sum_{q \neq 0} v_F \abs{q} \hat b_q^\dagger \hat b_q + (\text{a term for $q=0$})
\]
This is astonishing result of the bosonization method: the kinetic term can be approximated by a quadratic free-bosons hamiltonian. Any quartic fermion interaction term (as are two-body interactions) is density-quadratic and can be cast to an identical form.

\subsubsection*{Fermionic-bosonic correspondence}
At the very heart of the bosonization technique, lies a change of basis in operators space: the hamiltonian is mapped from a fermionic representation to a bosonic one, limitedly to the energy regime of our interest. In terms of the boson operators we shall express the fermion field operators,
\[
	\hat \psi_s (x) \equiv \frac{1}{\sqrt{L}} \sum_k e^{ikx} \hat c_k^{(s)}
\]
To derive the change of basis directly is non-trivial. However, it can be shown:
\[
	\comm{\hat \rho_s^\dagger (q)}{\hat \psi_s (x)} = -e^{iqx} \hat \psi_s (x)
\]
The above result is then used to extract the exact field representation in terms of density operators,
\[
	\hat \psi_s (x) = \hat U_s \exp\left\{ \sgn(s) \frac{2\pi}{L} \sum_q \frac{e^{iqx}}{q} \hat \rho_s (-q) \right\}
\]
where $\hat U_s$ is a so-called Klein-Haldane factor. The operator $\hat U_s$ suppresses a charge uniformly, and is inserted by hand to make the fermion-boson mapping coherent and bijective.

\subsubsection*{Field-theoretic representation of the free hamiltonian}

The final goal is to express the entire hamiltonian in terms of continuous bosonic fields. For now, define:
\[
\begin{aligned}
	\hat \phi(x) &\equiv - \colon \hat N \colon \frac{\pi x}{L} - \frac{i\pi}{L} \sum_{q \neq 0} \frac{e^{-\left( \frac{1}{2} \alpha \abs{q} + iqx \right)}}{q} \sum_{s \in \left\{\mathrm L, \mathrm R\right\}} \hat \rho_s^\dagger (q) &&\qquad\left( \alpha \to 0 \right) \\
	\hat \theta(x) &\equiv \colon \Delta \hat N \colon \frac{\pi x}{L} + \frac{i\pi}{L} \sum_{q \neq 0} \frac{e^{-\left( \frac{1}{2} \alpha \abs{q} + iqx \right)}}{q} \sum_{s \in \left\{\mathrm L, \mathrm R\right\}} \sgn(s) \hat \rho_s^\dagger (q) &&\qquad\left( \alpha \to 0 \right)
\end{aligned}
\]
where $\hat N = \hat N^{(\mathrm R)} + \hat N^{(\mathrm L)}$, $\Delta \hat N = \hat N^{(\mathrm R)} - \hat N^{(\mathrm L)}$ and $\alpha$ is a convergence cutoff to regularize the theory. Notice that the side-wise number operators appear normal-ordered, thus have finite matrix elements.
These field are defined like this for a reason: taking immediately the $\alpha \to 0$ limit and the $x$ derivative, we get
\beq\label{eq:fields-density-equations}
	\nabla \hat \phi(x) = - \pi \left[ \hat \rho_\mathrm{R}(x) + \hat \rho_\mathrm{L}(x) \right]
	\qquad
	\nabla \hat \theta(x) = \pi \left[ \hat \rho_\mathrm{R}(x) - \hat \rho_\mathrm{L}(x) \right]
\eeq
being the spatial density simply given by Fourier-transforming our $q$-wise density,
\[
	\hat \rho(x) = \frac{1}{L} \sum_q e^{-iqx} \hat \rho(q) = \frac{1}{L} \sum_q e^{-iqx} \sum_{s \in \left\{\mathrm L, \mathrm R\right\}} \hat \rho_s(q)
\]
Here, the second ``$=$'' sign is the passage where we actively switched to the Tomonaga-Luttinger model of Fig.~\ref{fig:bands}.
Then:
\[
\begin{aligned}
	- \frac{\nabla \hat \phi(x)}{\pi} \quad\to\quad &\text{particle density}\\
	\frac{\nabla \hat \theta(x)}{\pi} \quad\to\quad &\text{particle current}
\end{aligned}
\]
Clearly $\hat \rho_\mathrm{R} - \hat \rho_\mathrm{L}$ is the current operator: it just subtracts, point-wise, the left-going density from the right-going density. The remaining unbalance is the current by definition.

Let us go straight to the end: expressing the above fields in terms of boson operators it turns out that
\[
	\comm{\hat \phi(x)}{\frac{\nabla \hat \theta(y)}{\pi}} = i \delta(x-y)
\]
Thus, the fields $\hat \phi(x)$ and $\hat \Pi(x) \equiv \nabla \hat \phi(x)/\pi$ are bosonic and canonically conjugate.
Skipping some passages the reader can find in \cite{giamarchi2003quantum}, the hamiltonian is represented in field language as:
\beq\label{eq:free-field-hamiltonian}
	\hat H_0 \simeq \hat K_0 = \frac{1}{2\pi} \int_0^L dx \, v_F \left[ \left( \nabla \hat \phi(x) \right)^2 + \left( \nabla \hat \theta(x) \right)^2 \right]
\eeq
This is the very cornerstone of bosonization. This is the Klein-Gordon bosonic-massless hamiltonian. Apart from pure math, what we obtained is a consequence of the strict one-dimensional topology: in such low dimensionality the Fermi surface reduces to two points ($k=\pm k_F$), thus the only low-energy particle-hole excitations allowed (those collective excitations proper of a system of free fermions) either have a well defined momentum of $q \simeq 0$ or $q \simeq \pm 2k_F$. Low energy spectrum only exists strictly around these points.

Particle-hole excitations are always made of a combined creation and annihilation of fermions, thus intuitively remind of a ``bosonic character''. In order to interpret such excitations as bosons, however, they must be somewhat stable. This only happens in one dimension: here, particle-hole excitations \textit{are} emergent bosons. We do not enter in deep details here, recalling the main reference of this report \cite{giamarchi2003quantum} and its exceptional cover of the topic. To make the discussion here clearer, however, it must be cited that the reason for insurgence of boson fields is the fact that the use of a linear spectrum ensures independence of the particle-hole spectrum from the starting point on the (degenerated) Fermi surface, and thus lets us make the fermion-to-boson mapping.

\subsubsection*{Inserting interactions}

\begin{figure}
	\centering
	\subfloat[][$g_1 = V(q \approx \pm 2k_F)$ process.]{
		\begin{tikzpicture}[scale=0.6]
	\begin{axis}[
			axis lines=center,
			xlabel={$k$},
			ylabel={Energy},
			xlabel style={below},
			ylabel style={left},
			xtick=\empty,
			ytick=\empty,
			xmin=-3.5, xmax=3.5,
			ymin=-2.0, ymax=1.5
		]
		
		\addplot[domain=-3:3, smooth, color=tabblue]
			{-cos(deg(x))} node[below] {$\xi_k$};
		
		\addplot[domain=0.2:-2.8, color=tabgreen]
			{-1.57-x} node[right] {$\xi_k^{(\mathrm L)}$};
		
		\addplot[domain=-0.2:2.8, color=tabgreen]
			{-1.57+x} node[left] {$\xi_k^{(\mathrm R)}$};
		
		\draw[color=tabred,-stealth]
			(-1.37,-0.2) .. controls (-1,-0.2) and (1,0) .. (1.77,0.2) node[above left] {$g_1$};
		\draw[color=tabred,-stealth]
			(1.77,0.2) .. controls (1,0.2) and (-1,0) .. (-1.37,-0.2);
		
		\draw[color=tabred,dashed,-stealth]
			(1.37,-0.2) .. controls (1,-0.2) and (-1,0) .. (-1.77,0.2);
		\draw[color=tabred,dashed,-stealth]
			(-1.77,0.2) .. controls (-1,0.2) and (1,0) .. (1.37,-0.2);
		
	\end{axis}
\end{tikzpicture}
		\label{fig:g1-process}
	}
	\subfloat[][$g_2 = V(q \approx 0)$ process.]{
		\input{pictures/theory/g2-interactions}
		\label{fig:g2-process}
	}
	\subfloat[][$g_4 = V(q \approx 0)$ process.]{
		\input{pictures/theory/g4-interactions}
		\label{fig:g4-process}
	}
	\label{fig:g-processes}
	\caption{Diagrammatic sketch of the possible two-fermions interaction in the spinless scenario.}
\end{figure}

It is time to let in interactions. As we said, particle-hole excitations exchange a fermion from the Fermi sea with a hole from outside. Due to the strict topology of the Fermi surface, only three processes actually contribute -- namely $g_1$, $g_2$ and $g_4$, respectively in Figs.~\ref{fig:g1-process}-\ref{fig:g2-process}-\ref{fig:g4-process}. Note that, for spinless fermions, due to particles indistinguishability, actually $g_1$ and $g_2$ are the same process\footnote{
	We here skip an explanation about how we absorb $g_1$ inside $g_2$, a detail that will become clear in the spinful case.
}.

Now, consider a particle-hole symmetric interaction, \textit{quartic} in the fermionic operators,
\[
	V \sim c^\dagger c^\dagger c c
\]
as is for instance an $s$-wise spatial density-density interaction,
\[
	\hat V = \sum_{s_1 \in \left\{\mathrm L, \mathrm R\right\}} \sum_{s_2 \in \left\{\mathrm L, \mathrm R\right\}} \int_0^L dx_1 \int_0^L dx_2 \, V_{s_1 s_2}(x_1-x_2) \hat \rho_{s_1}(x_1) \hat \rho_{s_2} (x_2)
\]
coupling left-going and right-going fermions. We collect $g_4$ processes as those terms with $s_1 = s_2$ and $g_1$, $g_2$ processes as those terms with $s_1 \neq s_2$,
\[
	\underbrace{\hat \rho_\mathrm{R} (x_1) \hat \rho_\mathrm{R} (x_2) \quad \hat \rho_\mathrm{L} (x_1) \hat \rho_\mathrm{L} (x_2)}_{g_4} \qquad \underbrace{\hat \rho_\mathrm{R} (x_1) \hat \rho_\mathrm{L} (x_2) \quad \hat \rho_\mathrm{L} (x_1) \hat \rho_\mathrm{R} (x_2)}_{g_1 = g_2}
\]
At this point, we make an apparently heavy assumption we will heal later. Let us use for now a contact-like interaction,
\[
	\left[V(x-y)\right] = \frac{1}{2} \begin{bmatrix}
		g_4 & g_2 \\ g_2 & g_4
	\end{bmatrix} \delta(x-y)
	\qquad
	\text{with $g_2, g_4 \in \mathbb{R}$}
\]
(with a little notation abuse, we used the side indices $s_1$, $s_2$ as row-column indices) and let us analyze separately the contributions to the hamiltonian: $\hat V = \hat V_2 + \hat V_4$.

\paragraph{$g_4$ process.} this is the simpler case. The densities vertex contributions to $\hat K_0$ is simply
\[
	\hat V_4 = \frac{1}{2} g_4 \int_0^L dx \, \left[ \hat \rho_\mathrm{R} (x) \hat \rho_\mathrm{R} (x) + \hat \rho_\mathrm{L} (x) \hat \rho_\mathrm{L} (x) \right]
\]
Recalling Eq.~\eqref{eq:fields-density-equations}, we get immediately
\[
\begin{aligned}
	\hat V_4 &= g_4 \int_0^L dx \, \left[ \left( \frac{\nabla \hat \phi(x) - \nabla \hat \theta(x)}{2\pi} \right)^2 + \left( \frac{\nabla \hat \phi(x) + \nabla \hat \theta(x)}{2\pi} \right)^2 \right] \\
	&= \frac{g_4}{2\pi v_F} \times \frac{1}{2\pi} \int_0^L dx \, v_F \left[ \left( \nabla \hat \phi(x) \right)^2 + \left( \nabla \hat \theta(x) \right)^2 \right] \\
	&= \frac{g_4}{2\pi v_F} \hat K_0
\end{aligned}
\]
which is remarkable: considering this process, the hamiltonian looks like:
\[
	\hat K_0 + \hat V_4 + \hat V_2 = \frac{1}{2\pi} \int_0^L dx \, \underbrace{v_F \left( 1 + \frac{g_4}{2\pi v_F} \right)}_{u} \left[ \left( \nabla \hat \phi(x) \right)^2 + \left( \nabla \hat \theta(x) \right)^2 \right] + \hat V_2
\]
Now, $u$ is the \textbf{bosons velocity renormalized by $g_4$-like interactions}.

\paragraph{$g_2$ process.} In a very similar fashion, it is easy to obtain
\[
\begin{aligned}
	\hat V_2 &= \frac{1}{2} g_2 \int_0^L dx \, \left[ 2 \left( \frac{\nabla \hat \phi(x) - \nabla \hat \theta(x)}{2\pi} \right) \left( \frac{\nabla \hat \phi(x) + \nabla \hat \theta(x)}{2\pi} \right) \right] \\
	&= \frac{g_2}{2\pi v_F} \cdot \frac{1}{2\pi} \int_0^L dx \, v_F \left[ \left( \nabla \hat \phi(x) \right)^2 - \left( \nabla \hat \theta(x) \right)^2 \right]
\end{aligned}
\]
It is not so immediate to insert this term in the interacting hamiltonian. However, an elegant formulation exists involving two parameters $u$ and $K$:
\beq\label{eq:interacting-fields-hamiltonian}
	\hat K_0 + \hat V_4 + \hat V_2 = \frac{1}{2\pi} \int_0^L dx \, \left[ \frac{u}{K} \left( \nabla \hat \phi(x) \right)^2 + uK \left( \nabla \hat \theta(x) \right)^2 \right]
\eeq
trivially defined as
\[
	\frac{u}{K} \equiv 1 + \frac{g_4}{2\pi v_F} + \frac{g_2}{2\pi v_F}
	\qquad
	uK \equiv 1 + \frac{g_4}{2\pi v_F} - \frac{g_2}{2\pi v_F}
\]
a condition simultaneously satisfied by
\[
	u = v_F \sqrt{\left(1 + \frac{y_4}{2}\right)^2 - \left(\frac{y_2}{2}\right)^2}
	\qquad
	K = \sqrt{\frac{2 + y_4 - y_2}{2 + y_4 + y_2}}
	\qquad
	y_i \equiv \frac{g_i}{\pi v_F}
\]
This collection of equation is all we need to completely map a one-dimensional interacting fermionic problem into a renormalized free bosonic problem. Everything we have done hold for spinless fermions and contact interaction, but can be extended.

\subsubsection*{Spinless fermions observables}

The big, heavy (but wondrous) theoretical part is over: let's get operative. Our aim is to estimate the renormalized parameters $u$ and $K$. First, we need to understand how to get them out of some observables.

\paragraph{Charge compressibility.}

A very simple observable, capable of letting us estimate easily the ratio $u/K$, is charge compressibility. Let $\mu$ be the chemical potential,
\[
	\hat K \to \hat K - \mu \int_0^L dx \, \hat \rho(x)
\]
Following the convention of \citeauthor{giamarchi2003quantum}, we will define compressibility
\[
	\kappa \equiv \pdv{\rho}{\mu}
	\qquad
	\rho = \frac{1}{L} \int_0^L dx \, \ev{\hat \rho(x)}
\]
(notice that usually the definition above is completed by a prefactor $\rho^{-2}$, we omit). Using Eq.~\eqref{eq:fields-density-equations}\footnote{
	To be complete, we here are hiding a passage. In fact, $\pi \hat \rho = \pi \left[ \mathrm{RR} + \mathrm{LL} + \mathrm{RL} + \mathrm{LR} \right]$ (here we use the shorthand notation $\mathrm{s}_1 \mathrm{s}_2 = \hat \psi_{s_1}^\dagger \hat \psi_{s_2}$). Taking the average value, $\pi \ev{\hat \rho} = \pi \ev{ \mathrm{RR} + \mathrm{LL}} + \pi \ev{\mathrm{RL} + \mathrm{LR}}$, and it's evident by symmetry that $\ev{\mathrm{RL} + \mathrm{LR}} = 0$; which justifies our last line, since $\pi \ev{ \mathrm{RR} + \mathrm{LL}} = - \langle \nabla \hat \phi \rangle$.
},
\[
\begin{aligned}
	\pdv{\rho}{\mu} &= \pdv{}{\mu} \frac{1}{L} \int_0^L dx \, \ev{\hat \rho(x)} \\
	&= - \frac{1}{\pi L} \pdv{}{\mu} \int_0^L dx \, \ev{\nabla \hat \phi(x)}
\end{aligned}
\]

Now, consider the term we are adding to the hamiltonian: let us add a pure energy shift term (physically irrelevant) and manipulate the above expression a bit,
\[
\begin{aligned}
	- \mu \int_0^L dx \, \hat \rho(x) &= \frac{\mu}{\pi} \int_0^L dx \, \nabla \hat \phi(x) + \Delta \\
	&= \frac{1}{2\pi} \times 2 \int_0^L dx \, \frac{u}{K} \left( \mu \frac{K}{u} \right) \left( \nabla \hat \phi(x) \right) + \frac{1}{2\pi} \int_0^L dx \, \left( \mu \frac{K}{u} \right)^2 \\
\end{aligned}	
\]
It is now immediate to see that if we define
\[
	\hat \varphi(x) \equiv \hat \phi(x) + \mu \frac{K}{u} x
\]
we have:
\[
	\hat K - \mu \int_0^L dx \, \hat \rho(x) =  \frac{1}{2\pi} \int_0^L dx \, \left[ \frac{u}{K} \left( \vphantom{\hat\theta} \nabla \hat \varphi(x) \right)^2 + uK \left( \nabla \hat \theta(x) \right)^2 \right]
\]
an expression canonically equivalent to Eq.~\eqref{eq:interacting-fields-hamiltonian}. Now, for this new system the term $\nabla \hat \varphi$ represents charge density fluctuations. This implies that $\ev{\nabla \hat \varphi} = 0$ at any point. Then,
\[
	- \ev{\nabla \hat \phi(x)} = \mu \frac{K}{u}
\]
Finally:
\[
\begin{aligned}
	\pdv{\rho}{\mu}	&= - \frac{1}{\pi L} \pdv{}{\mu} \int_0^L dx \, \ev{\nabla \hat \phi(x)} \\
	&= \pdv{}{\mu} \mu \frac{K}{\pi u} \times \frac{1}{L} \int_0^L dx = \frac{K}{\pi u}
\end{aligned}
\]
Then, to measure the ratio $u/K$ we need to measure the quantity $(\pi \Delta \rho / \Delta \mu)^{-1}$ (times $\hbar$, to be dimensionally correct). This most certainly is a simple quantity to be measured by the means of a DMRG simulation.

\paragraph{Equal-time Green's function.}

The single-particle Green's function is defined in imaginary time, and for $s$-side fermions, as:
\[
	\mathcal{G}_s (x,\tau) \equiv - \ev{\mathrm{T}_\tau \left\{ \hat \psi_s (x,\tau) \hat \psi_s ^\dagger (0,0) \right\}}
\]
having we assumed in definition spacetime translational invariance, and being $\mathcal{T}_\tau$ the time-ordering operator. Let us take $\tau = 0^-$, thus keeping the order $\hat \psi \hat \psi^\dagger$ inside the expectation value. We follow here the lead of \cite{giamarchi2003quantum, Senechal:1999us}: the occupation factor $n(k)$,
\[
	n(k) \equiv \ev{\hat c_k^\dagger \hat c_k}
\]
is given by the Fourier transform of the equal time Green's function:
\[
	n(k) = \int_0^L dx \, e^{-ikx} \mathcal{G}_s (x,0^-)
\]
both for $s=\mathrm{R}, \mathrm{L}$ due to inversion symmetry. At zero temperature, we have the following algebraic dependence:
\[
	n(k) = n(k_F) - A \times \sgn(k-k_F) \abs{k-k_F}^\zeta
	\qquad
	\zeta \equiv \frac{1}{4} \left( K + \frac{1}{K} - 2 \right)
\]
with $A \in \mathbb{R}$. Then, a suitable way to extract the $K$ parameter on a lattice model simulation would be to perform the following computation:
\[
	n(k) \simeq \ev{ \mathrm{FT} \left\{ \hat c_j^\dagger \hat c_j \right\} } = \frac{1}{\sqrt{L}} \sum_{j=1}^L \ev{ \hat c_j^\dagger \hat c_j } e^{i k j}
\]
From this measure $\zeta$ can be extracted for $k<k_F$ and $k>k_F$, and from the latter $K$.

\paragraph{Density-density fluctuations.}

\todo

\paragraph{Things I would have done, if my code was better}

{\color{tabred} So far, I understood that my DMRG code is not feasible for: twisted boundary conditions (which is, magnetic flux threading a circular lattice Fermi-Hubbard model) and finite-low-temperature simulations. Unfortunately, I needed these to extract respectively: the charge stiffness $\mathcal{D} = uK$, and the specific heat $C_V = \frac{\pi LT}{3u}$. These measures, if possible, would have provided an estimation for $u$ and thus for $K$. Unluckily, we need to pass by correlations to get out $K$ -- a much more difficult road.}



\clearpage
\section{The Fermi-Hubbard model}

\begin{figure}
	\centering
	\subfloat[Lattice with null flux.]{
		\begin{tikzpicture}
	\foreach \i in {0,1,...,7}{
		\def\angle{45*\i}
		
		\draw[color=tabgreen]
			(\angle:2) -- ({\angle+45}:2);
			
		\node[anchor=center, color=tabgreen]
			at ({\angle+22.5}:2.2) {$t$};
		
		\filldraw[color=black]
			(\angle:2) circle (1pt);
	}
	
	\filldraw[color=black]
		(0:2) circle (1pt);
\end{tikzpicture}
		\label{subfig:lattice-ring}
	}
	\hfil
	\subfloat[Lattice with finite flux $\Phi$.]{
		\begin{tikzpicture}[
		decoration={
		markings,
		mark=at position 0.5 with {\arrow{stealth}}}
	]
	\foreach \i in {0,1,...,7}{
		\def\angle{45*\i}
		
		\draw[color=gray!50]
			(\angle:2) -- ({\angle+45}:2);
		
		\draw[color=tabgreen, postaction={decorate}]
			(\angle:2.05) -- ({\angle+45}:2.05);
		
		\node[anchor=center, color=tabgreen]
			at ({\angle+22.5}:2.2) {$T$};
		
		\draw[color=tabblue, postaction={decorate}]
			({\angle+45}:1.95) -- (\angle:1.95);
		
		\node[anchor=center, color=tabblue]
			at ({\angle+22.5}:1.45) {$T^*$};
		
		\filldraw[color=black]
			(\angle:2) circle (1pt);
	}
	
	\filldraw[color=black]
	(0:2) circle (1pt);
	
	\node[anchor=center] 
		at (0,0) {$\Phi$};
\end{tikzpicture}
		\label{subfig:lattice-ring-flux}
	}
	\caption{In Fig.~\ref{subfig:lattice-ring} a schematics of a $1D$ closed lattice is portrayed. The hopping amplitude $t$ is purely real, $t = \sgn(t) \abs{t}$. In Fig.~\ref{subfig:lattice-ring-flux} the same lattice is represented, but coupled to a finite threading flux $\Phi$ which can be absorbed via the pseudo-gauge transformation in Eq.~\todo. As a consequence, the hopping amplitude acquires a chirality which manifests in a non-null imaginary part, $t = \abs{t} e^{i\Phi/L}$.}
	\label{fig:lattice-rings}
\end{figure}

In this project we limit ourselves to a polarized (spinless fermions) system. Extension to a spinful system is possible and introduces some refinements in the general bosonization scheme, the most notable being the famous spin-charge separation. Let us take it easy: consider the following interacting hamiltonian:
\begin{equation}\label{eq:spinless-hamiltonian-pbc}
	\hat H \equiv t \sum_{\ev{j,k}} \left[
		\hat c_j^\dagger \hat c_k + \hat c_k^\dagger \hat c_j 
	\right] - V \sum_{\ev{j,k}} \hat n_j \hat n_k - \mu \sum_{j=1}^L \hat n_j
\end{equation}
defined on a closed $1D$ lattice ring, as in Fig.~\ref{subfig:lattice-ring}.
This is a simple nearest-neighbors (NN) interacting lattice hamiltonian with NN attraction $V$, chemical potential $\mu$ and hopping amplitude $t$,
\[
	t, V, \mu \in \mathbb{R}
\]

We will also be considering a magnetic flux $\Phi$ threading the ring and coupling to the charge degree of freedom. On a ring this pinned flux acts as a tangential vector potential, which is, a momentum offset; thus the correct way to absorb into our lattice framework this interaction is via the pseudo-gauge transformation
\[
	\hat c_j \to e^{-ij \phi} \hat c_j
	\qquad\qquad
	\hat c_j^\dagger \to e^{ij \phi} \hat c_j^\dagger
	\qquad\qquad
	\phi \equiv \frac{\Phi}{L}
\]
Incorporate the latter in the above hamiltonian: the hopping amplitude becomes complex (which is, chiral) $t = \abs{t} e^{\pm i\phi}$. We have, as in Fig.~\ref{subfig:lattice-ring-flux}
\begin{equation}\label{eq:spinless-hamiltonian-tbc}
	\hat H \equiv \abs{t} \sum_{j=1}^L \left[ 
		e^{-i\phi} \hat c_j^\dagger \hat c_{j+1} + e^{i\phi} \hat c_{j+1}^\dagger \hat c_j 
	\right] - V \sum_{j=1}^L \hat n_j \hat n_{j+1} - \mu \sum_{j=1}^L \hat n_j
\end{equation}
where a $\mod L$ operation is intended: $L+1 \leftrightarrow 1$. We want to indagate its ground-state properties. The relevant parameters will be the reduced hopping $t/V$ and chemical potential $\mu/V$.

\subsection{Jordan-Wigner mapping of the Heisenberg XXZ model}

The model presented above can be obtained rather easily through a Jordan-Wigner of the Heisenberg XXZ model in transverse field,
\begin{equation}\label{eq:xxz-hamiltonian}
	\hat H_\mathrm{XXZ} \equiv \sum_{\ev{j,k}} \left[
		J_{xy} \left( 
			\hat S_j^x \hat S_k^y + \hat S_k^y \hat S_j^x
		\right) + J_z \hat S_j^z \hat S_k^z
	\right] - h \sum_{j=1}^L \hat S_j^z 
\end{equation}
The Jordan-Wigner mapping, only feasible in one dimension due to sites ordering, is given by:
\[
	\hat S_j^+ \to \hat c_j^\dagger e^{i\pi \sum_{k < j} \hat c_k^\dagger \hat c_k}
	\qquad
	\hat S_j^- \to \hat c_j e^{- i\pi \sum_{k < j} \hat c_k^\dagger \hat c_k}
	\qquad
	\hat S_j^z \to \hat n_j - \frac{\mathbb{I}}{2}
\]
Consider first a spin chain with open boundary conditions: the mapping results in
\todo

\subsection{Bosonization of the model}

\todo

\subsection{Phenomenology of spin-charge separation}

\todo

\subsection{Adding one effective interaction: the Extended Fermi-Hubbard model}

\todo

\clearpage
\section{Algorithms and simulations}

This section is devoted to delineate the system properties we aim to simulate. The algorithm used is finite-size DMRG, implemented in the \href{https://docs.julialang.org/en/}{\texttt{Julia language}} via the well supported \href{https://itensor.github.io/ITensors.jl/stable/index.html}{\texttt{ITensors.jl}}, \href{https://itensor.github.io/ITensorMPS.jl/stable/}{\texttt{ITensorsMPS.jl}} packages.

\subsection{What I would have liked to do}

A good target is to extract the Bosonization parameters $u$ and $K$ for the spinless Fermi-Hubbard model of Eq.~\eqref{eq:spinless-hamiltonian-pbc}. As said in Sec.~\ref{subsubsec:spinless-fermions-observables}, for a spinless model the task can be easy enough by performing the calculation of the charge compressibility $\kappa$ and the charge stiffness $\mathcal{D}$. From Eq.~\eqref{eq:charge-compressibility} and \eqref{eq:charge-stiffness} respectively,
\begin{equation}\label{eq:charge-compressibility-stiffness-definitions}
	\kappa = \frac{K}{\pi u}
	\qquad\text{and}\qquad
	\mathcal{D} = uK
\end{equation}
which in turn implies
\begin{equation}\label{eq:u-K-formulas}
	u = \sqrt{\frac{\mathcal{D}}{\pi\kappa}}
	\qquad\text{and}\qquad
	K = \sqrt{\pi\kappa\mathcal{D}}
\end{equation}
Coherently with the definitions I used, the observables $\kappa$ and $\mathcal{D}$ were calculated as follows. Let me define:
\begin{equation}\label{eq:fixed-number-energy-definition}
	E_g \left[
		L,N,\eta;\frac{V}{t},\frac{\mu}{t}
	\right]
\end{equation}
as the ground-state energy for the model setup specified by its arguments $L$ (size), $N$ (number of particles), $\eta$ (adimensional magnetic flux) and parameters $V/t$ (reduced NN interaction) and $\mu/t$ (reduced chemical potential). From now on I omit $V/t$ and $\mu/t$ as explicit parameters. 

\paragraph{Charge compressibility.}
Charge compressibility is given by
\[
	\kappa^{-1} = L \pdv[2]{E}{N}
\]
which is well approximated at half-filling by
\begin{equation}\label{eq:charge-compressibility-approximation}
	\kappa_{1/2}(L) \equiv \left[
		\frac{E_g[L,L/2+2,0]+E_g[L,L/2-2,0]-2E_g[L,L/2,0]}{4}
	\right]^{-1}
\end{equation}
(usually one adds or removes $2$ particles in order to avoid even-odd effects). This strategy is good for mapping the compressibility in the canonical ensemble, for which the energy is minimized each time given a fixed particles number. For a mapping of compressibility over the $[V/t,\mu/t]$ evidently one needs to let the particles number vary in order to find the gran-canonical ground state. I adopted a rather rough but coherent strategy, approximated the compressibility via its finite-differences derivative formulation
\[
	\kappa \simeq \frac{\Delta \rho}{\Delta \mu} = \frac{1}{L} \frac{\Delta}{\Delta \mu} \langle \hat N \rangle
\]
where $\langle \hat N \rangle$ is the expected total particles number evaluated at two subsequent simulations with identical $V/t$ and chemical potential differing by $\Delta \mu$.

\paragraph{Charge stiffness.}
Similarly, charge stiffness is given by
\[
	\mathcal{D} = \pi L \pdv[2]{E}{\eta}
\]
which as well is approximated at half filling by
\begin{equation}\label{eq:charge-stiffness-approximation}
	\mathcal{D}_{1/2}(L) \equiv
	\frac{E_g[L,L/2,\delta\eta]+E_g[L,L/2,-\delta\eta]-2E_g[L,L/2,0]}{4(\delta\eta)^2}
\end{equation}
for a ``small'' flux variation $\delta\eta$. It is important to notice here that the charge compressibility is expected to vanish in gapped phases. The reason is simple: if $\partial_\mu \rho=0$, that means that shifting infinitesimally the chemical potential does not increase charge density -- there is no single-particle state that can accommodate additional particles. Thus, there is a gap. A simple and good signal that a phase has become gapless is the non-vanishing charge compressibility.

\subsection{What I actually did}

All I described in the above section is a rather good strategy, provided you can simulate a big long chubby chain with a lot of fermions. It turns out, taken into account my computational resources, the entire strategy turns out to be a little optimistic. It would have been wiser to understand my limits earlier, but that's how life goes, I guess.

\subsubsection*{Charge gaps}

For a model of the class of the spinless Fermi-Hubbard, Eq.~\eqref{eq:spinless-hamiltonian-pbc}, the chemical potential part amounts to a pure energy shift when working inside a fixed-number subspace of the many-body Hilbert space. Let $E_g[L,N,\eta]$ be the ground-state energy of Eq.~\eqref{eq:fixed-number-energy-definition} at fixed particle number $N$. Moreover, let me define
\[
	\Delta_\rho^{\pm M}[\mu] \equiv E_g[L, \rho L \pm M, \eta] - E_g[L, \rho L, \eta]
\]
being $\rho \equiv N/L$ the charge filling and $M \in \mathbb{N}$ a given number of particles ($+$ sign) or holes ($-$ sign) added as elementary excitations. To diagonalize the model \eqref{eq:spinless-hamiltonian-pbc} at fixed particles number means that the energy difference must depend on $\mu/t$ only by the total number of particles, while on $V/t$ in some complicated unspecified way. This implies
\[
	\Delta_\rho^{\pm M}[\mu] \equiv f\left(
		\rho, \frac{V}{t} 
	\right) \mp M \mu
	\quad\implies\quad
	f\left(
		\rho, \frac{V}{t} 
	\right) = \Delta_\rho^{\pm M}[\mu] \pm M\mu
\]
The parametric dependence of the $\Delta$s on $\mu$ was specified explicitly. The left-hand side is independent of $\mu$. Let me define $\mu_\rho^\pm$ as the chemical potential at which the gap closes -- which is, the set of points on the $[V/t,\mu/t]$ plane where to add particles or holes does not cost energy. Then, computing the above equation's right-hand side at zero chemical potential, it must be
\[
	\mu_\rho^{\pm M} = \mp \frac{1}{M} \Delta_\rho^{\pm M}[0]
\]
Now, recall the notation of Fig.~\ref{fig:expected-sfh-phase-diagram}: for the half-filling Mott insulating region, $\mathrm{MI}_{1/2}$, the top and bottom borders shall be given by
\[
	\mu_{1/2}^{\pm 1} = \mp \Delta_{1/2}^{\pm 1}[0]
	\qquad\text{and}\qquad
	\mu_{1/2}^{\pm 2} = \mp \frac{1}{2} \Delta_{1/2}^{\pm 2}[0]
\]
I reported both the expressions for $M=1$ and $M=2$: in our PBC-$\mathrm{XXZ}$ general scheme it is formally more correct to perform fixed-$N$ computations preserving the fermion number parity. For the unitary-filling region, $\mathrm{MI}_1$, of course it is not possible to add particles; as expected, just a $-$ phase boundary can be defined at unitary density
\[
	\mu_1^{-1} = -\Delta_1^{-1}[0]
	\qquad\text{and}\qquad
	\mu_1^{-2} = -\frac{1}{2} \Delta_1^{-2}[0]
\]
To compute $\Delta_\rho^{\pm M}[\mu]$ is just a matter of computing ground-state energies. This procedure allows for a simple estimation of the phase boundaries and, most importantly, provides insight on the finite-size effects.

\subsubsection*{Single-point characterization}

A good task is to characterize the three phases by choosing one precise set of parameters representative of each. Each choice $(V/t,\mu/t)$ describes a model whose solution is a given physical phase.

{\color{tabred} [Study of precise states (one for each phase): block density variance, compressibility and charge stiffness, plot of the correlation functions, bipartite entropy.]}

\subsubsection*{Observables heatmaps}
{\color{tabred} [Density heatmap and phase identification.]}

\clearpage
\section{Data analysis and results}

\todo

%\input{sections/appendix}

\printbibliography \nocite{*}

\end{document}
