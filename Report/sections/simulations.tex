\clearpage
\section{Algorithms and simulations}

This section is devoted to delineate the system properties we aim to simulate. The algorithm used is finite-size DMRG, implemented in the \href{https://docs.julialang.org/en/}{\texttt{Julia language}} via the well supported \href{https://itensor.github.io/ITensors.jl/stable/index.html}{\texttt{ITensors.jl}}, \href{https://itensor.github.io/ITensorMPS.jl/stable/}{\texttt{ITensorsMPS.jl}} packages.

\subsection{General strategy and target(s)}

My final aim is to extract the Bosonization parameters $u$ and $K$ for the spinless Fermi-Hubbard model of Eq.~\eqref{eq:spinless-hamiltonian-pbc}. As said in Sec.~\ref{subsubsec:spinless-fermions-observables}, for a spinless model the task can be easy enough by performing the calculation of the charge compressibility $\kappa$ and the charge stiffness $\mathcal{D}$. From Eq.~\eqref{eq:charge-compressibility} and \eqref{eq:charge-stiffness} respectively,
\begin{equation}\label{eq:charge-compressibility-stiffness-definitions}
	\kappa = \frac{K}{\pi u}
	\qquad\text{and}\qquad
	\mathcal{D} = uK
\end{equation}
which in turn implies
\begin{equation}\label{eq:u-K-formulas}
	u = \sqrt{\frac{\mathcal{D}}{\pi\kappa}}
	\qquad\text{and}\qquad
	K = \sqrt{\pi\kappa\mathcal{D}}
\end{equation}
Coherently with the definitions I used, the observables $\kappa$ and $\mathcal{D}$ were calculated as follows. Let me define:
\[
	E\left[
		L,N,\eta;\frac{V}{t},\frac{\mu}{t}
	\right]
\]
as the ground-state energy for the model setup specified by its arguments $L$ (size), $N$ (number of particles), $\eta$ (adimensional magnetic flux) and parameters $V/t$ (reduced NN interaction) and $\mu/t$ (reduced chemical potential). From now on I omit $V/t$ and $\mu/t$ as explicit parameters. 

\paragraph{Charge compressibility.}
Charge compressibility is given by
\[
	\kappa^{-1} = L \pdv[2]{E}{N}
\]
which is well approximated at half-filling by
\begin{equation}\label{eq:charge-compressibility-approximation}
	\kappa_{1/2}(L) \equiv \left[
		\frac{E[L,L/2+2,0]+E[L,L/2-2,0]-2E[L,L/2,0]}{4}
	\right]^{-1}
\end{equation}
(usually one adds or removes $2$ particles in order to avoid even-odd effects). This strategy is good for mapping the compressibility in the canonical ensemble, for which the energy is minimized each time given a fixed particles number. For a mapping of compressibility over the $[V/t,\mu/t]$ evidently one needs to let the particles number vary in order to find the gran-canonical ground state. I adopted a rather rough but coherent strategy, approximated the compressibility via its finite-differences derivative formulation
\[
	\kappa \simeq \frac{\Delta \rho}{\Delta \mu} = \frac{1}{L} \frac{\Delta}{\Delta \mu} \langle \hat N \rangle
\]
where $\langle \hat N \rangle$ is the expected total particles number evaluated at two subsequent simulations with identical $V/t$ and chemical potential differing by $\Delta \mu$.

\paragraph{Charge stiffness.}
Similarly, charge stiffness is given by
\[
	\mathcal{D} = \pi L \pdv[2]{E}{\eta}
\]
which as well is approximated at half filling by
\begin{equation}\label{eq:charge-stiffness-approximation}
	\mathcal{D}_{1/2}(L) \equiv
	\frac{E[L,L/2,\delta\eta]+E[L,L/2,-\delta\eta]-2E[L,L/2,0]}{4(\delta\eta)^2}
\end{equation}
for a ``small'' flux variation $\delta\eta$. It is important to notice here that the charge compressibility is expected to vanish in gapped phases. The reason is simple: if $\partial_\mu \rho=0$, that means that shifting infinitesimally the chemical potential does not increase charge density -- there is no single-particle state that can accommodate additional particles. Thus, there is a gap. A simple and good signal that a phase has become gapless is the non-vanishing charge compressibility.

%For a model defined on a lattice of lattice of $L \in \mathbb{N}$ sites with periodic boundary conditions, Born-Von Karman conditions apply, implying a limitation on the possible momenta:
%\[
%	k_n = \frac{2\pi}{L} n
%	\qquad\text{with}\qquad
%	n \in \mathbb{Z}
%\]
%This conditions implies $k_F = \pi$ at half-filling. Now, when imposing a finite flux through the ring -- a condition which, as discussed in Eq.~\eqref{eq:spinless-hamiltonian-tbc}, for arbitrary phase implies twisted boundary conditions with the hopping parameter transformation $t \to T = t^{i\phi}$ -- in order to maintain a coherent mapping with the $\mathrm{XXZ}$ model two things are needed: preservation of periodic boundary conditions (which is, twisting of $2\pi$ multiples or, equivalently, flux quantizatiom) and fermionic parity conservation. The latter is a built-in method of the \href{https://itensor.github.io/ITensorMPS.jl/stable/}{\texttt{ITensorsMPS.jl}} package. The former I implemented by simply choosing as hopping twisting phase the smallest non-zero crystal momentum allowed,
%\[
%	\phi = \frac{2\pi}{L}
%\]
%This condition is equivalent to requiring precisely one flux quantum to thread the ring.

\subsection{Finite-size DMRG}

\todo