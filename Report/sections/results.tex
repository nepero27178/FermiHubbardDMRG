\clearpage
\section{Data analysis and results}

\todo

\subsection{Ground-state phase physical properties}

As a first analysis, I considered two specific parametric configurations, whose Jordan-Wigner mapped ground-states are expected to lie respectively in the $\mathrm{XY}$ and Ising Ferromagnet phases,
\[
	\begin{aligned}
		[s_\mathrm{XY}] &\colon \left(
			\frac{h}{J_{xy}}, \frac{J_{z}}{J_{xy}}
		\right) = \left(
			-\frac{1}{2}, - \frac{1}{4}
		\right) &&\to &&\left(
			\frac{V}{t}, \frac{\mu}{t}
		\right) = \left(
			1, \frac{3}{2}
		\right) \\
		[s_\mathrm{IF}] &\colon \left(
			\frac{h}{J_{xy}}, \frac{J_{z}}{J_{xy}}
		\right) = \left(
			-1, \frac{1}{2}
		\right) &&\to &&\left(
			\frac{V}{t}, \frac{\mu}{t}
		\right) = \left(
			2, 1
		\right) \\
	\end{aligned}
\]
I shall indicate these configurations as indicated between square brackets.

{\color{red}\textbf{PER NON DIMENTICARMI DOVE SONO ARRIVATO} Continua con l'analisi degli stati qua sopra. Con potenziale chimico positivo dati i mappings pare che si possa solo vedere la transizione XY-FE. Fatta l'analisi di singoli stati, mappare i bordi di fase dal modello XXZ e fare una sweep orizzontale di compressibilità per vedere la fase gapless. Poi vedere se il calcolo del gap di carica produce dei boundareis sensati.}

\todo