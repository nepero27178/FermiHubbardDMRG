\section{Theoretical introduction to bosonization}

This first, vast section is devoted to the introduction of an extremely powerful technique in one-dimensional fermionic problems, namely, bosonization. It is widely based on the comprehensive work of \citeauthor{giamarchi2003quantum}, \cite{giamarchi2003quantum}. I won't get deep in the calculations neither in refined points about the method, being this last arbitrarily vast. The first part of this section deals with the spinless (a.k.a. polarized) case; the second part introduces the spin degree of freedom.

\subsection{Bosonization in a nutshell for spinless fermions}

\begin{figure}
	\centering
	\begin{tikzpicture}
	\begin{axis}[
			axis lines=center,
			xlabel={$k$},
			ylabel={Energy},
			xlabel style={below},
			ylabel style={left},
			xtick=\empty,
			ytick=\empty,
			xmin=-3.5, xmax=3.5,
			ymin=-1.0, ymax=2.5
		]
		\addplot[domain=-3:3, smooth, color=tabblue]
			{1-cos(deg(x))} node[below] {$\epsilon_k$};
		
		\addplot[domain=-3:3, color=tabred]
			{1} node[above] {$\epsilon_F$};
			
		\addplot[domain=0.2:-2.8, color=tabgreen]
			{-0.57-x} node[right] {$\epsilon_k^{(\mathrm L)}$};
		
		\addplot[domain=-0.2:2.8, color=tabgreen]
			{-0.57+x} node[left] {$\epsilon_k^{(\mathrm R)}$};
	\end{axis}
\end{tikzpicture}
	\caption{Sketch of the fermionic band $\epsilon_k$, the Fermi level $\epsilon_F$ and the two linear bands $\epsilon_k^{(\mathrm{L}/\mathrm{R})}$ used to approximate the original bands around the Fermi surface. The double-linear model is often referred to as the Tomonaga-Luttinger model.}
	\label{fig:bands}
\end{figure}

The key idea is simple: start from a conventional fermionic-metallic hamiltonian,
\[
	\hat H = \hat H_0 + \hat V = \sum_k \xi_k \hat c_k^\dagger \hat c_k + \hat V
\]
(I will leave the interaction unspecified, for a moment) where $\xi_k = \epsilon_k - \epsilon_F$ and I am using spinless fermions, for normal bands in ordinary fillings.
Consider Fig.~\ref{fig:bands}: the approximation in the above equation is exactly given by making the following assumption: since at low temperature (which, in metals, is a very broad definition) all the relevant Physics takes place at $\xi \sim 0$, and both the deep-down/far-away single-particle states do not contribute either due to Pauli pressure or state depletion, one can as well study the model:
\[
	\epsilon_k \to \left\{\epsilon_k^{(\mathrm L)}, \epsilon_k^{(\mathrm R)} \right\}
\]
Let $s$ be the side index, $s \in \left\{\mathrm L, \mathrm R\right\}$, with
\[
\sgn(s) = \begin{cases}
	+1 \quad \text{if} \quad s = \mathrm R \\
	-1 \quad \text{if} \quad s = \mathrm L
\end{cases}
\]
Then I may approximate around the Fermi surface (in one dimension degenerated in two points)
\[
	\hat H_0 \simeq \hat K_0 \equiv \sum_{s \in \left\{\mathrm L, \mathrm R\right\}} \sum_k \hbar v_F \left( \sgn(s) k -  k_F \right) \left[ \hat c_k^{(s)} \right]^\dagger \hat c_k^{(s)}
\]
being $k_F \equiv \sqrt{2m\epsilon_F/\hbar}$ the Fermi wavevector and $v_F \equiv \hbar k_F / m$. $\hat K_0$ is the Tomonaga-Luttinger model. In the following, I will set $\hbar=1$. Now, consider the side-wise density operators,
\[
	\hat \rho_q^{(s)} \equiv \sum_k  \left[ \hat c_{k+q}^{(s)} \right]^\dagger \hat c_k^{(s)}
\]
Let me use a slightly different, somewhat lighter notation:
\[
	\hat \rho_q^{(s)} \leftrightarrow \hat \rho_s (q)
\]
From now on, I will proceed only highlighting the important result in the bosonization procedure, since all the detailed derivation is included in \cite{giamarchi2003quantum}.

\subsubsection{Boson operators}

The pivotal result in the bosonization technique is the following:
\beq\label{eq:density-commutator}
	\comm{\hat \rho_s (q)}{\hat \rho_{s'} (-q')} = - \delta_{ss'} \delta_{qq'} \sgn(s) \frac{qL}{2\pi}
\eeq
where $L$ is the one-dimensional system length.
To get to this point, the very key passage is to employ the identity
\[
	\hat A \hat B = \colon \hat A \hat B \colon + \mel{\Omega}{\hat A \hat B}{\Omega}
\]
being $\hat A$, $\hat B$ two operators made of constructions/destructions, $\ket{\Omega}$ the generic many-body vacuum and $\colon \cdots \colon$ the normal ordering operation. Eq.~\eqref{eq:density-commutator} only holds if one uses this trick, which is, making a smart use of the infinite particle populations for the linearized model.

Now, Eq.~\eqref{eq:density-commutator} looks ``bosonic''. Notice that the left-side density operator vanishes identically for any $q>0$ on the ground-state Fermi sea $\ket{\Omega}$. This is because it would require to destroy a fermion at any given $k$ and creating one at $k+q$ -- but the monotonicity of the linear left band prevents from doing so, because states on the left get deeper and deeper and thus are already occupied. In formulas
\[
\begin{aligned}
	\hat \rho_\mathrm{L} (q>0) \ket{\Omega} &= 0 \\
	\hat \rho_\mathrm{R} (q<0) \ket{\Omega} &= 0
\end{aligned}
\]
Then, I define a boson operator with finite particle numbers as
\[
\begin{aligned}
	\hat b_q^\dagger &\equiv \sqrt{\frac{2\pi}{\abs{q}L}} \sum_{s \in \left\{\mathrm L, \mathrm R\right\}}  \left(\sgn(s) q\right) \hat \rho_s^\dagger (q) \\
	\hat b_q &\equiv \sqrt{\frac{2\pi}{\abs{q}L}} \sum_{s \in \left\{\mathrm L, \mathrm R\right\}} \mathrm{H} \left(\sgn(s) q\right) \hat \rho_s^\dagger (-q)
\end{aligned}
\]
(here I define the Heaviside step function $\mathrm{H}(x)$) which of course satisfy
\[
	\comm{\hat b_q}{\hat b_{q'}^\dagger} = \delta_{qq'}
\]

With a little patience, it can be shown that, taking $q \neq 0$,
\begin{equation}\label{eq:boson-hamiltonian-commutator}
	\comm{\hat b_q}{\hat K_0} = v_F \abs{q} \hat b_q
\end{equation}
Assuming the (operatorial) basis generated by the bosonic operators to be complete, then this equation completely defines $\hat K_0$. It must hold:
\[
	\hat K_0 = \sum_{q \neq 0} v_F \abs{q} \hat b_q^\dagger \hat b_q + (\text{a term for $q=0$})
\]
This is astonishing result of the bosonization method: the kinetic term can be approximated by a quadratic free-bosons hamiltonian. Any quartic fermion interaction term (as are two-body interactions) is density-quadratic and can be cast to an identical form.

\paragraph{Fermionic-bosonic correspondence}
At the very heart of the bosonization technique, lies a change of basis in operators space: the hamiltonian is mapped from a fermionic representation to a bosonic one, limitedly to the energy regime of interest. In terms of the boson operators I shall express the fermion field operators,
\[
	\hat \psi_s (x) \equiv \frac{1}{\sqrt{L}} \sum_k e^{ikx} \hat c_k^{(s)}
\]
which compose the total fermionic field just as
\[
	\hat \psi(x) = \hat \psi_\mathrm{R}(x) + \hat \psi_\mathrm{L}(x)
\]
I also define a shifted version of the fermionic field, just by centering the momenta around the Fermi points. This amounts to
\begin{equation}\label{eq:shifted-fermionic-fields-definition}
	\hat \psi_s (x) \equiv e^{i \sgn(s) k_F x} \hat \Psi_s (s)
\end{equation}

To derive the change of basis directly is non-trivial. However, it can be shown:
\[
	\comm{\hat \rho_s^\dagger (q)}{\hat \psi_s (x)} = -e^{iqx} \hat \psi_s (x)
\]
The above result is then used to extract the exact field representation in terms of density operators,
\begin{equation}\label{eq:fermionic-fields-density-expression}
	\hat \psi_s (x) = \hat U_s \exp\left\{ \sgn(s) \frac{2\pi}{L} \sum_q \frac{e^{iqx}}{q} \hat \rho_s (-q) \right\}
\end{equation}
where $\hat U_s$ is a so-called Klein-Haldane factor. The operator $\hat U_s$ suppresses a charge uniformly, and is inserted by hand to make the fermion-boson mapping coherent and bijective.

\subsubsection{Field-theoretic representation of the free hamiltonian}

The final goal is to express the entire hamiltonian in terms of continuous bosonic fields. For now, define:
\[
\begin{aligned}
	\hat \phi(x) &\equiv - \colon \hat N \colon \frac{\pi x}{L} - \frac{i\pi}{L} \sum_{q \neq 0} \frac{e^{-\left( \frac{1}{2} \alpha \abs{q} + iqx \right)}}{q} \sum_{s \in \left\{\mathrm L, \mathrm R\right\}} \hat \rho_s^\dagger (q) &&\qquad\left( \alpha \to 0 \right) \\
	\hat \theta(x) &\equiv \colon \Delta \hat N \colon \frac{\pi x}{L} + \frac{i\pi}{L} \sum_{q \neq 0} \frac{e^{-\left( \frac{1}{2} \alpha \abs{q} + iqx \right)}}{q} \sum_{s \in \left\{\mathrm L, \mathrm R\right\}} \sgn(s) \hat \rho_s^\dagger (q) &&\qquad\left( \alpha \to 0 \right)
\end{aligned}
\]
where $\hat N = \hat N^{(\mathrm R)} + \hat N^{(\mathrm L)}$, $\Delta \hat N = \hat N^{(\mathrm R)} - \hat N^{(\mathrm L)}$ and $\alpha$ is a convergence cutoff to regularize the theory. Notice that the side-wise number operators appear normal-ordered, thus have finite matrix elements.
These field are defined like this for a reason: taking immediately the $\alpha \to 0$ limit and the $x$ derivative,
\begin{equation}\label{eq:fields-density-equations}
	\nabla \hat \phi(x) = - \pi \left[ \hat \rho_\mathrm{R}(x) + \hat \rho_\mathrm{L}(x) \right]
	\qquad
	\nabla \hat \theta(x) = \pi \left[ \hat \rho_\mathrm{R}(x) - \hat \rho_\mathrm{L}(x) \right]
\end{equation}
being the spatial density simply given by Fourier-transforming our $q$-wise density,
\[
	\hat \rho(x) = \frac{1}{L} \sum_q e^{-iqx} \hat \rho(q) = \frac{1}{L} \sum_q e^{-iqx} \sum_{s \in \left\{\mathrm L, \mathrm R\right\}} \hat \rho_s(q)
\]
Here, the second ``$=$'' sign is the passage where I actively switched to the Tomonaga-Luttinger model of Fig.~\ref{fig:bands}.
Then:
\[
\begin{aligned}
	- \frac{\nabla \hat \phi(x)}{\pi} \quad\to\quad &\text{particle density, ``canonical position''}\\
	\frac{\nabla \hat \theta(x)}{\pi} \quad\to\quad &\text{particle $\mathrm{RL}$ unbalance, ``canonical momentum''}
\end{aligned}
\]
The difference $\hat \rho_\mathrm{R} - \hat \rho_\mathrm{L}$ is related to the current operator in one dimension: it just subtracts, point-wise, the left-going density from the right-going density.

For the sake of completeness, here I also report the expression for the fermionic fields in term of the bosonic ones:
\begin{equation}\label{eq:fermionic-fields-bosonic-fields-expression}
	\hat \psi_s (x) = \frac{\hat U_s}{\sqrt{2\pi\alpha}} \exp\left\{
		i \left(
			\sgn(s) k_F - \frac{\pi}{L}
		\right) x - i \left(
			\sgn(s) \hat \phi(x) + \hat \theta(x)
		\right)
	\right\}
	\qquad\left( \alpha \to 0 \right)
\end{equation}
This expression is valid in the thermodynamic limit. Let me go straight to the end: expressing the above fields in terms of boson operators it turns out that
\[
	\comm{\hat \phi(x)}{\frac{\nabla \hat \theta(y)}{\pi}} = i \delta(x-y)
\]
Thus, the fields $\hat \phi(x)$ and $\hat \Pi(x) \equiv \nabla \hat \phi(x)/\pi$ are bosonic and canonically conjugate.
Skipping some passages the reader can find in \cite[Chap.~2]{giamarchi2003quantum}, the hamiltonian is represented in field language as:
\begin{equation}\label{eq:free-field-hamiltonian}
	\hat H_0 \simeq \hat K_0 = \frac{1}{2\pi} \int_0^L dx \, v_F \left[ \left( \nabla \hat \phi(x) \right)^2 + \left( \nabla \hat \theta(x) \right)^2 \right]
\end{equation}
This is the very cornerstone of bosonization. This is the Klein-Gordon bosonic-massless hamiltonian. Apart from pure math, what we obtained is a consequence of the strict one-dimensional topology: in such low dimensionality the Fermi surface reduces to two points ($k=\pm k_F$), thus the only low-energy particle-hole excitations allowed (those collective excitations proper of a system of free fermions) either have a well defined momentum of $q \simeq 0$ or $q \simeq \pm 2k_F$. Low energy spectrum only exists strictly around these points.

Particle-hole excitations are always made of a combined creation and annihilation of fermions, thus intuitively remind of a ``bosonic character''. In order to interpret such excitations as bosons, however, they must be somewhat stable. This only happens in one dimension: here, particle-hole excitations \textit{are} emergent bosons. I won't enter in deep details here, recalling the main reference of this report \cite{giamarchi2003quantum} and its exceptional cover of the topic. To make the discussion here clearer, however, it must be cited that the reason for insurgence of boson fields is the fact that the use of a linear spectrum ensures independence of the particle-hole spectrum from the starting point on the (degenerated) Fermi surface, and thus makes the fermion-to-boson mapping possible.

\begin{table}
	\centering
	\begin{tabular}{r c c}
		Object && Density expression\\
		\midrule
		Fermionic field && $\displaystyle \hat \psi_s (x) = \hat U_s \exp\left\{ \sgn(s) \frac{2\pi}{L} \sum_q \frac{e^{iqx}}{q} \hat \rho_s (-q) \right\}$ \\
		&&\\
		Boson operator && $\displaystyle \hat b_q = \sqrt{\frac{2\pi}{\abs{q}L}} \sum_{s \in \left\{\mathrm L, \mathrm R\right\}} \theta \left(\sgn(s) q\right) \hat \rho_s^\dagger (-q)$ \\
		&&\\ 
		Boson fields gradients && $\displaystyle \begin{aligned}
			\nabla \hat \phi(x) &= - \pi \left[ \hat \rho_\mathrm{R}(x) + \hat \rho_\mathrm{L}(x) \right] \\ 
			\nabla \hat \theta(x) &= \pi \left[ \hat \rho_\mathrm{R}(x) - \hat \rho_\mathrm{L}(x) \right]
		\end{aligned}$
	\end{tabular}
	\caption{Summary of the relevant quantities of the bosonization scheme as expressions involving physical (side resolved) fermionic density.}
	\label{tab:bosonization-formulas}
\end{table}

As a final remark, notice that combining Eqns.~\eqref{eq:free-field-hamiltonian} and \eqref{eq:fields-density-equations}, the above hamiltonian reduces to
\begin{equation}\label{eq:free-field-hamiltonian-density-integral}
	\hat K_0 = \pi \int_0^L dx \, v_F \left[  
		\hat \rho_\mathrm{R}^2 (x) + \hat \rho_\mathrm{L}^2 (x)
	\right]
\end{equation}
a rewriting that will become useful later on. A prefactor $\hbar$ on the right side is to be reintroduced to be dimensionally correct.

\subsubsection{Inserting interactions}

\begin{figure}
	\centering
	\subfloat[][$g_1 = V(q \approx \pm 2k_F)$ process.]{
		\begin{tikzpicture}[scale=0.6]
	\begin{axis}[
			axis lines=center,
			xlabel={$k$},
			ylabel={Energy},
			xlabel style={below},
			ylabel style={left},
			xtick=\empty,
			ytick=\empty,
			xmin=-3.5, xmax=3.5,
			ymin=-2.0, ymax=1.5
		]
		
		\addplot[domain=-3:3, smooth, color=tabblue]
			{-cos(deg(x))} node[below] {$\xi_k$};
		
		\addplot[domain=0.2:-2.8, color=tabgreen]
			{-1.57-x} node[right] {$\xi_k^{(\mathrm L)}$};
		
		\addplot[domain=-0.2:2.8, color=tabgreen]
			{-1.57+x} node[left] {$\xi_k^{(\mathrm R)}$};
		
		\draw[color=tabred,-stealth]
			(-1.37,-0.2) .. controls (-1,-0.2) and (1,0) .. (1.77,0.2) node[above left] {$g_1$};
		\draw[color=tabred,-stealth]
			(1.77,0.2) .. controls (1,0.2) and (-1,0) .. (-1.37,-0.2);
		
		\draw[color=tabred,dashed,-stealth]
			(1.37,-0.2) .. controls (1,-0.2) and (-1,0) .. (-1.77,0.2);
		\draw[color=tabred,dashed,-stealth]
			(-1.77,0.2) .. controls (-1,0.2) and (1,0) .. (1.37,-0.2);
		
	\end{axis}
\end{tikzpicture}
		\label{fig:g1-process}
	}
	\subfloat[][$g_2 = V(q \approx 0)$ process.]{
		\begin{tikzpicture}[scale=0.75]
	\begin{axis}[
		axis lines=center,
		xlabel={$k$},
		ylabel={Energy},
		xlabel style={below},
		ylabel style={left},
		xtick=\empty,
		ytick=\empty,
		xmin=-3.5, xmax=3.5,
		ymin=-2.0, ymax=1.5
		]
		
		\addplot[domain=-3:3, smooth, color=tabblue]
			{-cos(deg(x))} node[below] {$\xi_k$};
		
		\addplot[domain=0.2:-2.8, color=tabgreen]
			{-1.57-x} node[right] {$\xi_k^{(\mathrm L)}$};
		
		\addplot[domain=-0.2:2.8, color=tabgreen]
			{-1.57+x} node[left] {$\xi_k^{(\mathrm R)}$};
		
		\draw[color=tabred,-stealth]
			(-1.37,-0.2) .. controls (-1.87, -0.1) .. (-1.77, 0.2);
		\draw[color=tabred,-stealth]
			(1.77,0.2) .. controls (1.87, -0.1) .. (1.37, -0.2) node[below right] {$g_2$};
			
		\draw[color=tabred,dashed,-stealth]
			(-1.77,0.2) .. controls (-1.27, 0.1) .. (-1.37,-0.2);
		\draw[color=tabred,dashed,-stealth]
			(1.37,-0.2) .. controls (1.27, 0.1) .. (1.77,0.2);
		
	\end{axis}
\end{tikzpicture}
		\label{fig:g2-process}
	}
	\subfloat[][$g_4 = V(q \approx 0)$ process.]{
		\begin{tikzpicture}[scale=0.6]
	\begin{axis}[
			axis lines=center,
			xlabel={$k$},
			ylabel={Energy},
			xlabel style={below},
			ylabel style={left},
			xtick=\empty,
			ytick=\empty,
			xmin=-3.5, xmax=3.5,
			ymin=-2.0, ymax=1.5
		]
		
		\addplot[domain=-3:3, smooth, color=tabblue]
			{-cos(deg(x))} node[below] {$\xi_k$};
		
		\addplot[domain=0.2:-2.8, color=tabgreen]
			{-1.57-x} node[right] {$\xi_k^{(\mathrm L)}$};
		
		\addplot[domain=-0.2:2.8, color=tabgreen]
			{-1.57+x} node[left] {$\xi_k^{(\mathrm R)}$};
		
		\draw[color=tabred,-stealth]
			(1.37,-0.2) .. controls (1.27, 0.1) .. (1.77, 0.2);
		\draw[color=tabred,-stealth]
			(1.77,0.2) .. controls (1.87, -0.1) .. (1.37, -0.2) node[below right] {$g_4$};
			
		\draw[color=tabred,dashed,-stealth]
			(-1.77,0.2) .. controls (-1.27, 0.1) .. (-1.37,-0.2);
		\draw[color=tabred,dashed,-stealth]
			(-1.37,-0.2) .. controls (-1.87, -0.1) .. (-1.77,0.2);
		
	\end{axis}
\end{tikzpicture}
		\label{fig:g4-process}
	}
	\caption{Diagrammatic sketch of the possible two-fermions interaction in the spinless scenario.}
	\label{fig:g-processes}
\end{figure}

It is time to let in interactions. As said, particle-hole excitations exchange a fermion from the Fermi sea with a hole from outside. Due to the strict topology of the Fermi surface, only three processes actually contribute -- namely $g_1$, $g_2$ and $g_4$, respectively in Figs.~\ref{fig:g1-process}-\ref{fig:g2-process}-\ref{fig:g4-process}. Note that, for spinless fermions, due to particles indistinguishability, actually $g_1$ and $g_2$ are the same process\footnote{
	I here skip an explanation about how to absorb $g_1$ inside $g_2$, a detail that will become clear in the spinful case.
}.

Now, consider a particle-hole symmetric interaction, \textit{quartic} in the fermionic operators,
\[
	V \sim c^\dagger c^\dagger c c
\]
as is for instance an $s$-wise spatial density-density interaction,
\[
	\hat V = \sum_{s_1 \in \left\{\mathrm L, \mathrm R\right\}} \sum_{s_2 \in \left\{\mathrm L, \mathrm R\right\}} \int_0^L dx_1 \int_0^L dx_2 \, V_{s_1 s_2}(x_1-x_2) \hat \rho_{s_1}(x_1) \hat \rho_{s_2} (x_2)
\]
coupling left-going and right-going fermions. I collect $g_4$ processes as those terms with $s_1 = s_2$ and $g_1$, $g_2$ processes as those terms with $s_1 \neq s_2$,
\[
	\underbrace{\hat \rho_\mathrm{R} (x_1) \hat \rho_\mathrm{R} (x_2) \quad \hat \rho_\mathrm{L} (x_1) \hat \rho_\mathrm{L} (x_2)}_{g_4} \qquad \underbrace{\hat \rho_\mathrm{R} (x_1) \hat \rho_\mathrm{L} (x_2) \quad \hat \rho_\mathrm{L} (x_1) \hat \rho_\mathrm{R} (x_2)}_{g_1 = g_2}
\]
At this point, I make an apparently heavy assumption I will heal later. Let me use for now a contact-like interaction,
\[
	\left[V(x-y)\right] = \frac{1}{2} \begin{bmatrix}
		g_4 & g_2 \\ g_2 & g_4
	\end{bmatrix} \delta(x-y)
	\qquad
	\text{with $g_2, g_4 \in \mathbb{R}$}
\]
(with a little notation abuse, I used the side indices $s_1$, $s_2$ as row-column indices) and let me analyze separately the contributions to the hamiltonian: $\hat V = \hat V_2 + \hat V_4$.

\paragraph{$g_4$ process.} this is the simpler case. The densities vertex contributions to $\hat K_0$ is simply
\[
	\hat V_4 = \frac{1}{2} g_4 \int_0^L dx \, \left[ \hat \rho_\mathrm{R} (x) \hat \rho_\mathrm{R} (x) + \hat \rho_\mathrm{L} (x) \hat \rho_\mathrm{L} (x) \right]
\]
Recalling Eq.~\eqref{eq:fields-density-equations},
\[
\begin{aligned}
	\hat V_4 &= g_4 \int_0^L dx \, \left[ \left( \frac{\nabla \hat \phi(x) - \nabla \hat \theta(x)}{2\pi} \right)^2 + \left( \frac{\nabla \hat \phi(x) + \nabla \hat \theta(x)}{2\pi} \right)^2 \right] \\
	&= \frac{g_4}{2\pi v_F} \times \frac{1}{2\pi} \int_0^L dx \, v_F \left[ \left( \nabla \hat \phi(x) \right)^2 + \left( \nabla \hat \theta(x) \right)^2 \right] \\
	&= \frac{g_4}{2\pi v_F} \hat K_0
\end{aligned}
\]
which is remarkable: considering this process, the hamiltonian looks like:
\[
	\hat K_0 + \hat V_4 + \hat V_2 = \frac{1}{2\pi} \int_0^L dx \, \underbrace{v_F \left( 1 + \frac{g_4}{2\pi v_F} \right)}_{u} \left[ \left( \nabla \hat \phi(x) \right)^2 + \left( \nabla \hat \theta(x) \right)^2 \right] + \hat V_2
\]
Now, $u$ is the \textbf{bosons velocity renormalized by $g_4$-like interactions}.

\paragraph{$g_2$ process.} In a very similar fashion, it is easy to obtain
\[
\begin{aligned}
	\hat V_2 &= \frac{1}{2} g_2 \int_0^L dx \, \left[ 2 \left( \frac{\nabla \hat \phi(x) - \nabla \hat \theta(x)}{2\pi} \right) \left( \frac{\nabla \hat \phi(x) + \nabla \hat \theta(x)}{2\pi} \right) \right] \\
	&= \frac{g_2}{2\pi v_F} \cdot \frac{1}{2\pi} \int_0^L dx \, v_F \left[ \left( \nabla \hat \phi(x) \right)^2 - \left( \nabla \hat \theta(x) \right)^2 \right]
\end{aligned}
\]
It is not so immediate to insert this term in the interacting hamiltonian. However, an elegant formulation exists involving two parameters $u$ and $K$:
\begin{equation}\label{eq:interacting-fields-hamiltonian}
	\hat K_0 + \hat V_4 + \hat V_2 = \frac{1}{2\pi} \int_0^L dx \, \left[ \frac{u}{K} \left( \nabla \hat \phi(x) \right)^2 + uK \left( \nabla \hat \theta(x) \right)^2 \right]
\end{equation}
trivially defined as
\[
	\frac{u}{K} \equiv 1 + \frac{g_4}{2\pi v_F} + \frac{g_2}{2\pi v_F}
	\qquad
	uK \equiv 1 + \frac{g_4}{2\pi v_F} - \frac{g_2}{2\pi v_F}
\]
a condition simultaneously satisfied by
\[
	u = v_F \sqrt{\left(1 + \frac{y_4}{2}\right)^2 - \left(\frac{y_2}{2}\right)^2}
	\qquad
	K = \sqrt{\frac{2 + y_4 - y_2}{2 + y_4 + y_2}}
	\qquad
	y_i \equiv \frac{g_i}{\pi v_F}
\]
This collection of equation is all I need to completely map a one-dimensional interacting fermionic problem into a renormalized free bosonic problem. Everything I have done hold for spinless fermions and contact interaction, but can be extended.

\subsubsection{The euclidean action}

I here briefly sketch the derivation of the bosonized euclidean action. Starting from the hamiltonian density,
\[
	\hat{\mathcal{K}}[\phi, \Pi] = \frac{1}{2\pi} \left[
			\frac{u}{K} \left( \nabla \hat \phi(x) \right)^2 + uK \left( \pi \hat \Pi(x) \right)^2
	\right]
\]
the euclidean action is immediately recovered by Legendre-transforming $\hat{\mathcal{K}}$ in imaginary time,
\[
	\begin{aligned}
		\mathcal{L}[\phi, \Pi] &= i \Pi \pdv{\tau} \phi - \mathcal{K} [\phi, \Pi] \\
		&= \frac{1}{2\pi} \left[
			2i \nabla \theta \pdv{\tau} \phi
			- \frac{u}{K} \left( \nabla \phi \right)^2 + uK \left( \nabla \theta \right)^2
		\right]
	\end{aligned}
\]
Now the calculation gets a little intricate, and I will skip it. The key point is to recognize that, being $L$ the lagrangian,
\[
	S[\phi, \Pi] \equiv \int_0^\beta d\tau \, L[\phi, \Pi] = \int_0^\beta d\tau \int_0^L dx \,  \mathcal{L}[\phi, \Pi]
\]
and $\mathcal{Z}$ the partition function
\[
	\mathcal{Z} \equiv \int \mathcal{D}[\phi] \mathcal{D}[\Pi] e^{-S[\phi, \Pi] /\hbar}
\]
one is able to complete the square for the $\nabla \theta$ part appearing in $\mathcal{L}$ and reduce the $\Pi$ part of the above integral to a gaussian form. The same trick holds for any $\Pi$-independent observable we may want to average. The final, effective $\phi$-action is just
\begin{equation}\label{eq:phi-action}
	S_\phi \equiv \frac{1}{2\pi} \int_0^\beta d\tau \int_0^L dx \, \left[
		\frac{1}{uK} \left(
			\vphantom{A^A}
			\partial_\tau \phi(x,\tau)
		\right)^2 + \frac{u}{K} \left(
			\vphantom{A^A}
			\nabla \phi(x,\tau) 
		\right)^2
	\right]
\end{equation}
which, exponentiated, is the path integral statistical weight.

\subsubsection{Spinless fermions observables}\label{subsubsec:spinless-fermions-observables}

The big, heavy (but wondrous) theoretical part is over: let's get operative. My aim is to estimate the renormalized parameters $u$ and $K$. First, it must be understood how to get them out of some observables.

\paragraph{Charge compressibility.}

A very simple observable, usable for estimating easily the ratio $u/K$, is charge compressibility. Let $\mu$ be the chemical potential,
\[
	\hat K \to \hat K - \mu \int_0^L dx \, \hat \rho(x)
\]
Following the convention of \citeauthor{giamarchi2003quantum}, I will define compressibility as
\[
	\kappa \equiv \pdv{\rho}{\mu}
	\qquad
	\rho = \frac{1}{L} \int_0^L dx \, \ev{\hat \rho(x)}
\]
(notice that usually the definition above is completed by a prefactor $\rho^{-2}$, I omit). Using Eq.~\eqref{eq:fields-density-equations}\footnote{
	To be complete, I here am hiding a passage. In fact, $\pi \hat \rho = \pi \left[ \mathrm{RR} + \mathrm{LL} + \mathrm{RL} + \mathrm{LR} \right]$ (here I use the shorthand notation $\mathrm{s}_1 \mathrm{s}_2 = \hat \psi_{s_1}^\dagger \hat \psi_{s_2}$). Taking the average value, $\pi \ev{\hat \rho} = \pi \ev{ \mathrm{RR} + \mathrm{LL}} + \pi \ev{\mathrm{RL} + \mathrm{LR}}$, and it's evident by symmetry that $\ev{\mathrm{RL} + \mathrm{LR}} = 0$; which justifies the last line, since $\pi \ev{ \mathrm{RR} + \mathrm{LL}} = - \langle \nabla \hat \phi \rangle$.
},
\[
\begin{aligned}
	\pdv{\rho}{\mu} &= \pdv{}{\mu} \frac{1}{L} \int_0^L dx \, \ev{\hat \rho(x)} \\
	&= - \frac{1}{\pi L} \pdv{}{\mu} \int_0^L dx \, \ev{\nabla \hat \phi(x)}
\end{aligned}
\]

Now, consider the term I am adding to the hamiltonian: let me add a pure energy shift term (physically irrelevant) and manipulate the above expression a bit,
\[
\begin{aligned}
	- \mu \int_0^L dx \, \hat \rho(x) &= \frac{\mu}{\pi} \int_0^L dx \, \nabla \hat \phi(x) + \Delta \\
	&= \frac{1}{2\pi} \times 2 \int_0^L dx \, \frac{u}{K} \left( \mu \frac{K}{u} \right) \left( \nabla \hat \phi(x) \right) + \underbrace{
		\frac{1}{2\pi} \int_0^L dx \, \left( \mu \frac{K}{u} \right)^2
	}_{\Delta} \\
\end{aligned}	
\]
It is now immediate to see that if I define
\[
	\hat \varphi(x) \equiv \hat \phi(x) + \mu \frac{K}{u} x
\]
we have:
\[
	\hat K - \mu \int_0^L dx \, \hat \rho(x) =  \frac{1}{2\pi} \int_0^L dx \, \left[ \frac{u}{K} \left( \vphantom{\hat\theta} \nabla \hat \varphi(x) \right)^2 + uK \left( \nabla \hat \theta(x) \right)^2 \right]
\]
an expression canonically equivalent to Eq.~\eqref{eq:interacting-fields-hamiltonian}. Now, for this new system the term $\nabla \hat \varphi$ represents charge density fluctuations. This implies that $\ev{\nabla \hat \varphi} = 0$ at any point. Then,
\[
	- \ev{\nabla \hat \phi(x)} = \mu \frac{K}{u}
\]
Finally:
\begin{equation}
	\begin{aligned}
		\pdv{\rho}{\mu}	&= - \frac{1}{\pi L} \pdv{}{\mu} \int_0^L dx \, \ev{\nabla \hat \phi(x)} \\
		&= \pdv{}{\mu} \mu \frac{K}{\pi u} \times \frac{1}{L} \int_0^L dx = \frac{K}{\pi u} \label{eq:charge-compressibility}
	\end{aligned}
\end{equation}
Then, to measure the ratio $u/K$ I need to measure the quantity $(\pi \Delta \rho / \Delta \mu)^{-1}$ (times $\hbar$, to be dimensionally correct). This most certainly is a simple quantity to be measured by the means of a DMRG simulation.

\paragraph{Charge stiffness}

\begin{figure}
	\centering
	\begin{tikzpicture}[
	scale=1.5,
	decoration={
		markings,
		mark=at position 0.6 with {\arrow{stealth}}}
	]
	
	\draw[color=black, stealth-stealth]
		(0,0) -- (2,0) node[anchor=south, midway]
			{$L/2\pi$};
	\draw[color=black]
		(0,0) ellipse (2 and 0.5);
	\draw[color=tabred, postaction={decorate}] 
		(0,0) ellipse (2.2 and 0.6) node[anchor=north east, yshift=-25, xshift=-30]
			{$\displaystyle A = \Phi/L$};
	\fill[color=white]
		(-0.05, 0.45) rectangle (0.05, 0.65);
	\draw[color=tabblue, -stealth] 
		(0,-0.45) -- (0,1) node[anchor=south east]
			{$\displaystyle B = 4\pi \Phi/L^2$};
	\draw[color=tabblue]
		(0,-0.65) -- (0,-1);
\end{tikzpicture}
	\caption{Schematics of a static flux $\Phi$ pinning through a ring of circumference $L$.}
	\label{fig:flux-through-ring}
\end{figure}

Another observable rather easy to compute numerically is the charge stiffness; which is basically the tendency of the system to respond to an external charge-coupling field. Now, assuming a closed-chain topology (as in Fig.~\ref{fig:flux-through-ring}), a flux $\Phi$ threading the ring accounts for a static vector potential
\[
	A = \frac{\Phi}{L}
\]
I define the unit flux as:	$\Phi_0 \equiv h/e$. The overall effect on the fermionic system is an overall momentum shift originated by the covariant derivative formulation,
\[
	k \to k - \frac{e}{\hbar} A = k - \frac{2\pi}{L} \frac{\Phi}{\Phi_0}
\]
Let me define the shift angle:
\[
	\eta \equiv 2\pi \frac{\Phi}{\Phi_0}
\]
Finally, absorbing the latter as a gauge transformation, this translates in the presence of twisted boundary conditions by an angle $2\pi \Phi/\Phi_0$ in the wavefunction. Now, given the band $\xi_k[\Phi]$, the current density is simply
\[
	\begin{aligned}
		J &= \frac{1}{L} \sum_{k \colon \xi_k \le 0} \frac{1}{\hbar} \pdv{}{k} \xi_k[\Phi] \bigg|_\Phi \\
		&= J_0 +  \frac{1}{L\hbar} \sum_{k \colon \xi_k \le 0} L \pdv{}{\eta} \xi_k[\Phi] \bigg|_\Phi \\
		&= \frac{1}{\hbar} \pdv{\eta} E[\Phi]
	\end{aligned}
\]
In the second passage I have isolated the contribution to the derivative given from the free system $J_0$, null by symmetry, and in the last the ground-state energy $E[\Phi]$ was recognized as the sum of all single-particle occupied states.

Now, charge stiffness is simply the zero-flux current response:
\[
	\mathcal{D} = \pdv{J}{\Phi} \bigg|_{\Phi=0}
\]
As for the compressibility, I adopt a slightly different (and charge-neutral) definition,
\begin{equation}\label{eq:charge-stiffness-definition}
	\mathcal{D} \equiv \frac{\pi L}{e} \pdv{J}{\Phi} \bigg|_{\Phi=0}
\end{equation}
Inserting the previous result,
\[
	\mathcal{D} = \frac{\pi L}{e} \frac{\Phi_0}{2\pi} \pdv{J}{\eta} \bigg|_{\Phi=0} = \pi L \pdv[2]{\eta} E[\Phi] \bigg|_{\Phi=0}
\]
The second derivative of the ground-state energy, taken with respect to the twisting angle, is -- apart from some factors -- the charge stiffness.

I now need to link all of this with the bosonization scheme. The procedure is identical to the one carried out for the charge compressibility in last paragraph:
\begin{enumerate}
	\item Include the minimally coupled interaction in the euclidean effective action of Eq.~\eqref{eq:phi-action},
	\[
		S_\phi \to S_\phi - \int dx \, J(x) A(x) = S_\phi - \frac{\Phi}{L} \int dx \, J(x)
	\]
	\item The charge density current is one dimension is obtained easily from the continuity equation:
	\[
		\partial_t \rho + \nabla j = 0
		\qquad\implies\qquad
		j = \frac{1}{\pi} \partial_t \phi
	\]
	having used $\rho = - \nabla \phi/\pi$ and ignored boundary terms.
	\item A constant added to the action does not change its variational properties. Thus, defining
	\[
		\varphi \equiv \phi - uK \frac{\Phi}{L} \tau
	\]
	neither the time nor the spatial part get affected by the transformation, and the action for a flux-free system is recovered.
	\item Since for the flux-free system the induced current is zero,
	\[
		J = \frac{1}{\pi} \ev{\partial_\tau \phi} = uK \frac{\Phi}{\pi L}
	\]
\end{enumerate}
Recalling the definition of $\mathcal{D}$ of Eq.~\eqref{eq:charge-stiffness-definition}, finally
\begin{equation}\label{eq:charge-stiffness}
	\mathcal{D} = uK
\end{equation}
which is the second relation I needed in order to determine $u$ and $K$. The entirety of this derivation could have been worked out analogously by expressing the current in terms of the conjugate momentum field $\Pi$ and completing the square directly in the hamiltonian.

\paragraph{Equal-time Green's function.}

The single-particle Green's function is defined in imaginary time, and for $s$-side fermions, as:
\[
	\mathcal{G}_s (x,\tau) \equiv - \ev{\mathrm{T}_\tau \left\{ \hat \psi_s (x,\tau) \hat \psi_s ^\dagger (0,0) \right\}}
\]
having we assumed in definition spacetime translational invariance, and being $\mathcal{T}_\tau$ the time-ordering operator. Let me take $\tau = 0^-$, thus keeping the order $\hat \psi \hat \psi^\dagger$ inside the expectation value. I (surprisingly) follow here the lead of \cite{giamarchi2003quantum, senechal1999introduction}: the occupation factor $n(k)$,
\[
	n(k) \equiv \ev{\hat c_k^\dagger \hat c_k}
\]
is given by the Fourier transform of the equal time Green's function:
\[
	n(k) = \int_0^L dx \, e^{-ikx} \mathcal{G}_s (x,0^-)
\]
both for $s=\mathrm{R}, \mathrm{L}$ due to inversion symmetry. At zero temperature, the following algebraic dependence holds:
\[
	n(k) = n(k_F) - A \times \sgn(k-k_F) \abs{k-k_F}^\zeta
	\qquad
	\zeta \equiv \frac{1}{4} \left( K + \frac{1}{K} - 2 \right)
\]
with $A \in \mathbb{R}$. Then, a suitable way to extract the $K$ parameter on a lattice model simulation would be to perform the following computation:
\[
	n(k) \simeq \ev{ \mathrm{FT} \left\{ \hat c_j^\dagger \hat c_j \right\} } = \frac{1}{\sqrt{L}} \sum_{j=1}^L \ev{ \hat c_j^\dagger \hat c_j } e^{i k j}
\]
From this measure $\zeta$ can be extracted for $k<k_F$ and $k>k_F$, and from the latter $K$.

\paragraph{Equal-time density-density correlations.}

A very important feature of the bosonization scheme is the possibility of computing first-order analytical expressions for the correlations of observables. Take density-density correlations,
\[
	\mathcal{C}_\mathrm{CDW}(s) = \ev{\hat \rho(s) \hat \rho(0)}
	\quad\text{for}\quad
	s = (x,u\tau)
\]
To treat this expression in a field-theoretical fashion, I express the density operator as
\[
	\begin{aligned}
		\hat \rho(s) &= \colon \left[
			\hat \psi_\mathrm{R}^\dagger(s) + \hat \psi_\mathrm{L}^\dagger(s)
		\right] \left[
			\hat \psi_\mathrm{R}(s) + \hat \psi_\mathrm{L}(s)
		\right] \colon \\
		&= - \frac{1}{\pi} \nabla \hat \phi(s) + \left[
			\hat \psi_\mathrm{R}^\dagger(s) \hat \psi_\mathrm{L}(s) + \hat \psi_\mathrm{L}^\dagger(s) \hat \psi_\mathrm{R}(s) 
		\right] \\
		&= - \frac{1}{\pi} \nabla \hat \phi(s) + \frac{1}{2\pi\alpha} \left[
			e^{i 2k_F x} e^{-i 2 \hat \phi(s)}
			+ \mathrm{h.c.}
		\right]
	\end{aligned}
\]
Recall, $\alpha$ is a cutoff I introduced to stabilize the theory. In particular, $\alpha$ has the dimensions of a length and the momentum integral is cut at momenta larger than $1/\alpha$. I avoid entering in the detailed derivation: what is important, is that we can safely assume $\alpha \to 0$, at least for this correlation function \cite{haldane1981luttinger}. Let $y_\alpha$ be
\[
	y_\alpha \equiv u\tau + \alpha \sgn(\tau)
\]
The dominant behavior for $\mathcal{C}_\mathrm{CDW}(s)$ is given by
\[
	\mathcal{C}_\mathrm{CDW}(s) \simeq \frac{K}{2\pi^2} \frac{y_\alpha^2 - x^2}{(x^2 + y_\alpha^2)^2} + \frac{2}{(2\pi\alpha)^2} \cos\left(2k_F x\right) \left(
		\frac{\alpha}{\sqrt{x^2 + (u\tau)^2}}
	\right)^{2K}
	\quad\text{for}\quad \sqrt{x^2 + (u\tau)^2} \gg \alpha
\]
which, for equal-time measurement ($\tau=0$) reduces to
\begin{equation}\label{eq:density-density-correlation-power-law}
	\mathcal{C}_\mathrm{CDW}(x) \simeq - \frac{K}{2\pi^2 x^2} +  \frac{2}{(2\pi\alpha)^2} \cos\left(2k_F x\right) \left(
		\frac{\alpha}{x}
	\right)^{2K}
\end{equation}
As \citeauthor{giamarchi2003quantum} notes \cite[Sec.~2.2.2]{giamarchi2003quantum}, the first term reproduces a standard Fermi liquid correlation, decaying as $x^{-2}$ with an amplitude renormalized by interactions. The second term is highly unusual and has two remarkable properties: an amplitude modulation of wavevector $2k_F$ (a property also present in Fermi liquids, signaling the presence of a Fermi sea underneath the interactions) \textit{plus} a power-law decay whose strength is determined by interactions.

Finally, notice that for a $1$D lattice of lattice spacing $a$ at half-filling, the Fermi wavevector is given by $k_F = \pi/2a$. Since each site coordinate is $x_r=ra$, one has
\[
	\mathcal{C}_\mathrm{CDW}(r) \simeq - \frac{K}{2\pi^2 (ra)^2} +  \frac{2 (-1)^r }{(2\pi\alpha)^2} \left(
	\frac{\alpha}{ra}
	\right)^{2K}
	\quad\text{for}\quad
	r \in \mathbb{Z} \mod L
\]
The oscillatory character of the correlation function with the site index parity reflects the fact that, at half-filling, charge-density waves (CDW) excitations follow have wavelength twice the lattice spacing. At unitary filling, such oscillatory behavior vanishes due to the halving of the Fermi wavevector.

\paragraph{Superconducting correlations.}

The superconducting phase is gapless and governed by a tendency to fermion pairing. Consider the NN-pairing operator
\[
	\hat \psi(x) \hat \psi(x)
\]
Of course, defined like this, such an operator identically vanishes: two fermions cannot be annihilated simultaneously at $x$. Consider the slight extension
\[
	\hat{\mathcal{O}}_\mathrm{SU}(x) \equiv \hat \psi (x+\delta x) \hat \psi (x)
\]
For a lattice the spatial shift can be taken to be $\delta x = a$, the lattice spacing. This operator describes for a lattice the tendency to pairing \textit{locally}. Now, when expanded in terms of Right and Left fields,
\[
	\begin{aligned}
		\hat \psi (x+a) \hat \psi(x) &= \left[
			\hat \psi_\mathrm{R} (x+a) +
			\hat \psi_\mathrm{L} (x+a)
		\right] \left[
			\hat \psi_\mathrm{R} (x) +
			\hat \psi_\mathrm{L} (x)
		\right] \\
		&= (\mathrm{RR}) + (\mathrm{RL}) + (\mathrm{LR}) + (\mathrm{LL})
	\end{aligned}
\]
the $(\mathrm{RR})$ and $(\mathrm{LL})$ terms are strongly suppressed in the $a \to 0$ term due to Pauli principle. The terms $(\mathrm{RL})$ and $(\mathrm{LR})$ are dominant. Since
\[
	\begin{aligned}
		(\mathrm{RL}) + (\mathrm{LR}) &= \hat \psi_\mathrm{R} (x+a) \hat \psi_\mathrm{L} (x) + \hat \psi_\mathrm{L} (x+a) \hat \psi_\mathrm{R} (x) \\
		&= e^{ik_F a} \hat \Psi_\mathrm{R} (x+a) \hat \Psi_\mathrm{L} (x) + e^{-ik_F a} \hat \Psi_\mathrm{L} (x+a) \hat \Psi_\mathrm{R} (x) \\
		&= e^{ik_F a} \hat \Psi_\mathrm{R} (x+a) \hat \Psi_\mathrm{L} (x) - e^{-ik_F a} \hat \Psi_\mathrm{R} (x) \hat \Psi_\mathrm{L} (x+a)\\
		&\simeq 2i \sin(k_F a) \hat \Psi_\mathrm{R} (x) \hat \Psi_\mathrm{L} (x) \\
		&\simeq \frac{i \sin(k_F a)}{\pi\alpha} e^{-i \left[ \hat \phi(x) + \hat \theta(x) \right] } e^{i \left[ \hat \phi(x) - \hat \theta(x) \right]} \\
		(\text{half filling})&\simeq \frac{i}{\pi\alpha} e^{-2i\hat\theta(x)}
	\end{aligned}
\]
I used a sequence of approximations: the continuum limit was taken, $a \to 0$, using the slow variation of the shifted fields $\hat \Psi_s$; then the fields $\hat \Psi_s$ were written explicitly in terms of their bosonic counterpart (ignoring the Klein factors), and finally I merged the two exponential operators ignoring the correction terms.

The superconducting correlator is defined as
\[
	\mathcal{C}_\mathrm{SU}(x) = \ev{\hat{\mathcal{O}}_\mathrm{SU}^\dagger(x) \hat{\mathcal{O}}_\mathrm{SU}(x)}
\]
It can be shown that using the previous result gives:
\begin{equation}\label{eq:superconducting-correlation-power-law}
	\mathcal{C}_\mathrm{SU}(x) = \frac{1}{(\pi\alpha)^2} \left(
		\frac{\alpha}{x}
	\right)^{1/2K}
\end{equation}
The most important detail to notice with respect to the $\mathrm{CDW}$ correlator of Eq.~\eqref{eq:density-density-correlation-power-law} is the inverted power-law dependence.