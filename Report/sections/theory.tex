\section{Theoretical introduction}

\todo

\subsection{Bosonization in a nutshell}

\begin{figure}
	\centering
	\begin{tikzpicture}
	\begin{axis}[
			axis lines=center,
			xlabel={$k$},
			ylabel={Energy},
			xlabel style={below},
			ylabel style={left},
			xtick=\empty,
			ytick=\empty,
			xmin=-3.5, xmax=3.5,
			ymin=-1.0, ymax=2.5
		]
		\addplot[domain=-3:3, smooth, color=tabblue]
			{1-cos(deg(x))} node[below] {$\epsilon_k$};
		
		\addplot[domain=-3:3, color=tabred]
			{1} node[above] {$\epsilon_F$};
			
		\addplot[domain=0.2:-2.8, color=tabgreen]
			{-0.57-x} node[right] {$\epsilon_k^{(\mathrm L)}$};
		
		\addplot[domain=-0.2:2.8, color=tabgreen]
			{-0.57+x} node[left] {$\epsilon_k^{(\mathrm R)}$};
	\end{axis}
\end{tikzpicture}
	\caption{Sketch of the fermionic band $\epsilon_k$, the Fermi level $\epsilon_F$ and the two linear bands $\epsilon_k^{(\mathrm{L}/\mathrm{R})}$ used to approximate the original bands around the Fermi surface.}
	\label{fig:bands}
\end{figure}

This sections is intended as a brief sketch and discussion of the powerful \textbf{bosonization} method in one dimension. This section is widely based on the comprehensive work of \citeauthor{giamarchi2003quantum}, \cite{giamarchi2003quantum}. The key idea is simple: start from a conventional fermionic-metallic hamiltonian,
\[
	\hat H = \hat H_0 + \hat V = \sum_k \xi_k \hat c_k^\dagger \hat c_k + \hat V
\]
(we will leave the interaction unspecified, for a moment) where $\xi_k = \epsilon_k - \epsilon_F$ and we are using spinless fermions, for normal bands in ordinary fillings.
Consider Fig.~\ref{fig:bands}: the approximation in the above equation is exactly given by making the following assumption: since at low temperature (which, in metals, is a very broad definition) all the relevant Physics takes place at $\xi \sim 0$, and both the deep-down/far-away single-particle states do not contribute either due to Pauli pressure or state depletion, we can as well study the model:
\[
	\epsilon_k \to \left\{\epsilon_k^{(\mathrm L)}, \epsilon_k^{(\mathrm R)} \right\}
\]
Let $s$ be the side index, $s \in \left\{\mathrm L, \mathrm R\right\}$, with
\[
\sgn(s) = \begin{cases}
	+1 \quad \text{if} \quad s = \mathrm R \\
	-1 \quad \text{if} \quad s = \mathrm L
\end{cases}
\]
Then we may approximate around the Fermi surface (in one dimension degenerated in two points)
\[
	\hat H_0 \simeq \hat K_0 \equiv \sum_{s \in \left\{\mathrm L, \mathrm R\right\}} \sum_k v_F \left( \sgn(s) k -  k_F \right) \left[ \hat c_k^{(s)} \right]^\dagger \hat c_k^{(s)}
\]
being $k_F \equiv \sqrt{2m\epsilon_F}$ the Fermi wavevector and $v_F \equiv k_F / m$. Now, consider the side-wise density operators,
\[
	\hat \rho_q^{(s)} \equiv \sum_k  \left[ \hat c_{k+q}^{(s)} \right]^\dagger \hat c_k^{(s)}
\]
Let us use a slightly different, somewhat lighter notation:
\[
	\hat \rho_q^{(s)} \leftrightarrow \hat \rho_s (q)
\]
From now on, we will proceed only highlighting the important result in the bosonization procedure, since all the detailed derivation is included in \cite{giamarchi2003quantum}.

\subsubsection*{Boson operators}

The pivotal result in the bosonization technique is the following:
\beq\label{eq:density-commutator}
	\comm{\hat \rho_s (q)}{\hat \rho_{s'} (-q')} = - \delta_{ss'} \delta_{qq'} \sgn(s) \frac{qL}{2\pi}
\eeq
To get to this point, the very key passage is to employ the identity
\[
	\hat A \hat B = \colon \hat A \hat B \colon + \mel{\Omega}{\hat A \hat B}{\Omega}
\]
being $\hat A$, $\hat B$ two operators made of constructions/destructions, $\ket{\Omega}$ the generic many-body vacuum and $\colon \cdots \colon$ the normal ordering operation. Eq.~\eqref{eq:density-commutator} only holds if we use this trick, which is, if we make a smart use of the infinite particle populations for the linearized model.

Now, Eq.~\eqref{eq:density-commutator} looks ``bosonic''. Notice that the left-side density operator vanishes identically for any $q>0$ on the ground-state Fermi sea $\ket{\Omega}$. This is because it would require to destroy a fermion at any given $k$ and creating one at $k+q$ -- but the monotonicity of the linear left band prevents from doing so, because states on the left get deeper and deeper and thus are already occupied. In formulas
\[
\begin{aligned}
	\hat \rho_\mathrm{L} (q>0) \ket{\Omega} &= 0 \\
	\hat \rho_\mathrm{R} (q<0) \ket{\Omega} &= 0
\end{aligned}
\]
Then, we can define boson operator with finite particle numbers as
\[
\begin{aligned}
	\hat b_q^\dagger &\equiv \sqrt{\frac{2\pi}{\abs{q}L}} \sum_{s \in \left\{\mathrm L, \mathrm R\right\}} \theta \left(\sgn(s) q\right) \hat \rho_s^\dagger (q) \\
	\hat b_q &\equiv \sqrt{\frac{2\pi}{\abs{q}L}} \sum_{s \in \left\{\mathrm L, \mathrm R\right\}} \theta \left(\sgn(s) q\right) \hat \rho_s^\dagger (-q)
\end{aligned}
\]
which of course satisfy
\[
	\comm{\hat b_q}{\hat b_{q'}^\dagger} = \delta_{qq'}
\]

With a little patience, it can be shown that, taking $q \neq 0$,
\beq\label{eq:boson-hamiltonian-commutator}
	\comm{\hat b_q}{\hat K_0} = v_F \abs{q} \hat b_q
\eeq
Assuming the (operatorial) basis generated by the bosonic operators to be complete, then this equation completely defines $\hat K_0$. It must hold:
\[
	\hat K_0 = \sum_{q \neq 0} v_F \abs{q} \hat b_q^\dagger \hat b_q + (\text{a term for $q=0$})
\]
This is astonishing result of the bosonization method: the kinetic term can be approximated by a quadratic free-bosons hamiltonian. Any quartic fermion interaction term (as are two-body interactions) is density-quadratic and can be cast to an identical form.

\subsubsection*{Fermionic-bosonic correspondence}
At the very heart of the bosonization technique, lies a change of basis in operators space: the hamiltonian is mapped from a fermionic representation to a bosonic one, limitedly to the energy regime of our interest. In terms of the boson operators we shall express the fermion field operators,
\[
	\hat \psi_s (x) \equiv \frac{1}{\sqrt{L}} \sum_k e^{ikx} \hat c_k^{(s)}
\]
To derive the change of basis directly is non-trivial. However, it can be shown:
\[
	\comm{\hat \rho_s^\dagger (q)}{\hat \psi_s (x)} = -e^{iqx} \hat \psi_s (x)
\]
The above result is then used to extract the exact field representation in terms of density operators,
\[
	\hat \psi_s (x) = \hat U_s \exp\left\{ \sgn(s) \frac{2\pi}{L} \sum_q \frac{e^{iqx}}{q} \hat \rho_s (-q) \right\}
\]
where $\hat U_s$ is a so-called Klein factor. The operator $\hat U_s$ suppresses a charge uniformly, and is inserted by hand to make the fermion-boson mapping coherent and bijective.

\subsubsection*{Field-theoretic representation of the hamiltonian}

The final goal is to express the entire hamiltonian in terms of continuous bosonic fields. For now, define:
\[
\begin{aligned}
	\hat \phi(x) &\equiv - \colon \hat N \colon \frac{\pi x}{L} - \frac{i\pi}{L} \sum_{q \neq 0} \frac{e^{-\left( \frac{1}{2} \alpha \abs{q} + iqx \right)}}{q} \sum_{s \in \left\{\mathrm L, \mathrm R\right\}} \hat \rho_s^\dagger (q) \;\Bigg|_{\alpha \to 0} \\
	&= - \colon \hat N \colon \frac{\pi x}{L} - \frac{i\pi}{L} \sum_{q \neq 0} \sqrt{\frac{\abs{q} L}{2\pi}} \frac{e^{-\left( \frac{1}{2} \alpha \abs{q} + iqx \right)}}{q} \left( \hat b_q^\dagger + \hat b_{-q} \right)\Bigg|_{\alpha \to 0} \\
	\hat \theta(x) &\equiv \colon \Delta \hat N \colon \frac{\pi x}{L} + \frac{i\pi}{L} \sum_{q \neq 0} \frac{e^{-\left( \frac{1}{2} \alpha \abs{q} + iqx \right)}}{q} \sum_{s \in \left\{\mathrm L, \mathrm R\right\}} \sgn(s) \hat \rho_s^\dagger (q) \;\Bigg|_{\alpha \to 0} \\
	&= \colon \Delta \hat N \colon \frac{\pi x}{L} + \frac{i\pi}{L} \sum_{q \neq 0} \sqrt{\frac{\abs{q} L}{2\pi}} \frac{e^{-\left( \frac{1}{2} \alpha \abs{q} + iqx \right)}}{\abs{q}} \left( \hat b_q^\dagger - \hat b_{-q} \right)\Bigg|_{\alpha \to 0} \\
\end{aligned}
\]
where $\hat N = \hat N^{(\mathrm R)} + \hat N^{(\mathrm L)}$, $\Delta \hat N = \hat N^{(\mathrm R)} - \hat N^{(\mathrm L)}$ and $\alpha$ is a convergence cutoff to regularize the theory. Notice that the side-wise number operators appear normal-ordered, thus have finite matrix elements.

Let us go straight to the end: expressing the above fields in terms of boson operators it turns out that
\[
	\comm{\hat \phi(x)}{\frac{\grad \hat \theta(y)}{\pi}} = i \delta(x-y)
\]
Thus, the fields $\hat \phi(x)$ and $\hat \Pi(x) \equiv \grad \hat \phi(x)/\pi$ are bosonic and canonically conjugate.
Skipping some passages the reader can find in \cite{giamarchi2003quantum}, the hamiltonian is represented in field language as:
\beq\label{eq:free-field-hamiltonian}
	\hat H_0 \simeq \hat K_0 = \frac{1}{2\pi} \int_\mathbb{R} dx \, v_F \left[ \left( \grad \hat \phi(x) \right)^2 + \left( \grad \hat \theta(x) \right)^2 \right]
\eeq
This is the very cornerstone of bosonization. 

\subsubsection*{Inserting interactions}

\begin{figure}
	\centering
	\subfloat[][$g_2 = V(q \approx 0)$ process.]{
		\begin{tikzpicture}[scale=0.75]
	\begin{axis}[
		axis lines=center,
		xlabel={$k$},
		ylabel={Energy},
		xlabel style={below},
		ylabel style={left},
		xtick=\empty,
		ytick=\empty,
		xmin=-3.5, xmax=3.5,
		ymin=-2.0, ymax=1.5
		]
		
		\addplot[domain=-3:3, smooth, color=tabblue]
			{-cos(deg(x))} node[below] {$\xi_k$};
		
		\addplot[domain=0.2:-2.8, color=tabgreen]
			{-1.57-x} node[right] {$\xi_k^{(\mathrm L)}$};
		
		\addplot[domain=-0.2:2.8, color=tabgreen]
			{-1.57+x} node[left] {$\xi_k^{(\mathrm R)}$};
		
		\draw[color=tabred,-stealth]
			(-1.37,-0.2) .. controls (-1.87, -0.1) .. (-1.77, 0.2);
		\draw[color=tabred,-stealth]
			(1.77,0.2) .. controls (1.87, -0.1) .. (1.37, -0.2) node[below right] {$g_2$};
			
		\draw[color=tabred,dashed,-stealth]
			(-1.77,0.2) .. controls (-1.27, 0.1) .. (-1.37,-0.2);
		\draw[color=tabred,dashed,-stealth]
			(1.37,-0.2) .. controls (1.27, 0.1) .. (1.77,0.2);
		
	\end{axis}
\end{tikzpicture}
		\label{fig:g2-process}
	}
	\subfloat[][$g_4 = V(q \approx 0)$ process.]{
		\begin{tikzpicture}[scale=0.6]
	\begin{axis}[
			axis lines=center,
			xlabel={$k$},
			ylabel={Energy},
			xlabel style={below},
			ylabel style={left},
			xtick=\empty,
			ytick=\empty,
			xmin=-3.5, xmax=3.5,
			ymin=-2.0, ymax=1.5
		]
		
		\addplot[domain=-3:3, smooth, color=tabblue]
			{-cos(deg(x))} node[below] {$\xi_k$};
		
		\addplot[domain=0.2:-2.8, color=tabgreen]
			{-1.57-x} node[right] {$\xi_k^{(\mathrm L)}$};
		
		\addplot[domain=-0.2:2.8, color=tabgreen]
			{-1.57+x} node[left] {$\xi_k^{(\mathrm R)}$};
		
		\draw[color=tabred,-stealth]
			(1.37,-0.2) .. controls (1.27, 0.1) .. (1.77, 0.2);
		\draw[color=tabred,-stealth]
			(1.77,0.2) .. controls (1.87, -0.1) .. (1.37, -0.2) node[below right] {$g_4$};
			
		\draw[color=tabred,dashed,-stealth]
			(-1.77,0.2) .. controls (-1.27, 0.1) .. (-1.37,-0.2);
		\draw[color=tabred,dashed,-stealth]
			(-1.37,-0.2) .. controls (-1.87, -0.1) .. (-1.77,0.2);
		
	\end{axis}
\end{tikzpicture}
		\label{fig:g4-process}
	}
	\label{fig:g-processes}
	\caption{Diagrammatic sketch of the possible two-fermions interaction in the spinless scenario.}
\end{figure}

\subsubsection*{Renormalization interpretation}

\subsection{The Fermi-Hubbard model}

\todo

\subsection{Adding one effective interaction}

\todo