\clearpage
\section{The Fermi-Hubbard model}

\begin{figure}
	\centering
	\subfloat[Lattice with null flux.]{
		\begin{tikzpicture}
	\foreach \i in {0,1,...,7}{
		\def\angle{45*\i}
		
		\draw[color=tabgreen]
			(\angle:2) -- ({\angle+45}:2);
			
		\node[anchor=center, color=tabgreen]
			at ({\angle+22.5}:2.2) {$t$};
		
		\filldraw[color=black]
			(\angle:2) circle (1pt);
	}
	
	\filldraw[color=black]
		(0:2) circle (1pt);
\end{tikzpicture}
		\label{subfig:lattice-ring}
	}
	\hfil
	\subfloat[Lattice with finite flux $\Phi$.]{
		\begin{tikzpicture}[
		decoration={
		markings,
		mark=at position 0.5 with {\arrow{stealth}}}
	]
	\foreach \i in {0,1,...,7}{
		\def\angle{45*\i}
		
		\draw[color=gray!50]
			(\angle:2) -- ({\angle+45}:2);
		
		\draw[color=tabgreen, , postaction={decorate}]
			(\angle:2.05) -- ({\angle+45}:2.05);
		
		\node[anchor=center, color=tabgreen]
			at ({\angle+22.5}:2.2) {$T$};
		
		\draw[color=tabblue, , postaction={decorate}]
			({\angle+45}:1.95) -- (\angle:1.95);
		
		\node[anchor=center, color=tabblue]
			at ({\angle+22.5}:1.45) {$T^*$};
		
		\filldraw[color=black]
			(\angle:2) circle (1pt);
	}
	
	\filldraw[color=black]
	(0:2) circle (1pt);
	
	\node[anchor=center] 
		at (0,0) {$\Phi$};
\end{tikzpicture}
		\label{subfig:lattice-ring-flux}
	}
	\caption{In Fig.~\ref{subfig:lattice-ring} a schematics of a $1D$ closed lattice is portrayed. The hopping amplitude $t$ is purely real, $t = \sgn(t) \abs{t}$. In Fig.~\ref{subfig:lattice-ring-flux} the same lattice is represented, but coupled to a finite threading flux $\Phi$ which can be absorbed via the pseudo-gauge transformation in Eq.~\todo. As a consequence, the hopping amplitude acquires a chirality which manifests in a non-null imaginary part, $T = t e^{i\Phi/L}$.}
	\label{fig:lattice-rings}
\end{figure}

In this project we limit ourselves to a polarized (spinless fermions) system. Extension to a spinful system is possible and introduces some refinements in the general bosonization scheme, the most notable being the famous spin-charge separation. Let us take it easy: consider the following interacting hamiltonian:
\begin{equation}\label{eq:spinless-hamiltonian-pbc}
	\hat H \equiv t \sum_{\ev{j,k}} \left[
		\hat c_j^\dagger \hat c_k + \hat c_k^\dagger \hat c_j 
	\right] + V \sum_{\ev{j,k}} \hat n_j \hat n_k - \mu \sum_{j=1}^L \hat n_j
\end{equation}
defined on a closed $1D$ lattice ring, as in Fig.~\ref{subfig:lattice-ring}.
This is a simple nearest-neighbors (NN) interacting lattice hamiltonian with NN interaction $V$, chemical potential $\mu$ and hopping amplitude $t$,
\[
	t, V, \mu \in \mathbb{R}
\]

We will also be considering a magnetic flux $\Phi$ threading the ring and coupling to the charge degree of freedom. On a ring this pinned flux acts as a tangential vector potential, which is, a momentum offset; thus the correct way to absorb into our lattice framework this interaction is via the pseudo-gauge transformation
\[
	\hat c_j \to e^{-ij \phi} \hat c_j
	\qquad\qquad
	\hat c_j^\dagger \to e^{ij \phi} \hat c_j^\dagger
	\qquad\qquad
	\phi \equiv \frac{\Phi}{L}
\]
Incorporate the latter in the above hamiltonian: the hopping amplitude becomes complex (which is, chiral) $t \to T \equiv t e^{i\phi}$, with $t, \phi \in \mathbb{R}$. We have, as in Fig.~\ref{subfig:lattice-ring-flux}
\begin{equation}\label{eq:spinless-hamiltonian-tbc}
	\hat H \equiv t \sum_{j=1}^L \left[ 
		e^{-i\phi} \hat c_j^\dagger \hat c_{j+1} + e^{i\phi} \hat c_{j+1}^\dagger \hat c_j 
	\right] + V \sum_{j=1}^L \hat n_j \hat n_{j+1} - \mu \sum_{j=1}^L \hat n_j
\end{equation}
where a $\mod L$ operation is intended: $L+1 \leftrightarrow 1$. We want to indagate its ground-state properties. The relevant parameters will be the reduced hopping $t/V$ and chemical potential $\mu/V$.

\subsection{Jordan-Wigner mapping of the Heisenberg XXZ model}

{\color{tabred}[Insert here: schematic of the spin chain mapped onto the fermionic chain]}

The model presented above can be obtained rather easily through a Jordan-Wigner of the Heisenberg XXZ model in transverse field,
\begin{equation}\label{eq:xxz-hamiltonian}
	\hat H_\mathrm{XXZ} \equiv \sum_{\ev{j,k}} \left[
		J_{xy} \left( 
			\hat S_j^x \hat S_k^x + \hat S_k^y \hat S_j^y
		\right) + J_z \hat S_j^z \hat S_k^z
	\right] - h \sum_{j=1}^L \hat S_j^z 
\end{equation}
The Jordan-Wigner mapping, only feasible in one dimension due to sites ordering, is given by:
\[
	\hat S_j^+ \to \hat c_j^\dagger e^{i\pi \sum_{k < j} \hat c_k^\dagger \hat c_k}
	\qquad
	\hat S_j^- \to \hat c_j e^{- i\pi \sum_{k < j} \hat c_k^\dagger \hat c_k}
	\qquad
	\hat S_j^z \to \hat n_j - \frac{\mathbb{I}}{2}
\]
Notice the appearance of the Jordan string,
\[
	e^{i\pi \sum_{k < j} \hat c_k^\dagger \hat c_k} = (-1)^{\zeta_j}
	\qquad
	\zeta_j \equiv \sum_{k < j} \hat n_k
\]
Essentially, the above string counts the fermions \textit{before} the site in question and gives back a factor $+1$ for even number, $-1$ for odd number. This works for open-end chains, where the concept of \textit{before} is actually well-defined. It is straightforward to see:
\[
	\begin{aligned}
		S_j^+ \hat S_{j+1}^- &\to (-1)^{\zeta_j + \zeta_{j+1}} \hat c_j^\dagger \hat c_{j+1} = (-1)^{n_j} \hat c_j^\dagger \hat c_{j+1} = c_j^\dagger \hat c_{j+1} \\
		S_j^- \hat S_{j+1}^+ &\to (-1)^{\zeta_j + \zeta_{j+1}} \hat c_j \hat c_{j+1}^\dagger = (-1)^{n_j+1} \hat c_{j+1}^\dagger \hat c_j = c_{j+1}^\dagger \hat c_j \\
	\end{aligned}
\]
having we used $\zeta_j + \zeta_{j+1} = 2 \zeta_j + n_j$.
Let us take a $1\mathrm{D}$ spin chain with open boundary conditions (OBC). The transformation gives
\[
	\begin{aligned}
		\hat H_\mathrm{XXZ} &\equiv \sum_{j=1}^{L-1} \left[
			\frac{J_{xy}}{2} \left( 
				\hat S_j^+ \hat S_{j+1}^- + \hat S_j^- \hat S_{j+1}^+
			\right) + J_z \hat S_j^z \hat S_{j+1}^z
		\right] - h \sum_{j=1}^L \hat S_j^z \\
		&= \sum_{j=1}^{L-1} \left[
			\frac{J_{xy}}{2} \left(
				\hat c_j^\dagger \hat c_{j+1} + \hat c_{j+1}^\dagger \hat c_j
			\right) + J_z \left(
				\hat n_j - \frac{\mathbb{I}}{2}
			\right) \left(
				\hat n_{j+1} - \frac{\mathbb{I}}{2}
			\right)
		\right] - h \sum_{j=1}^L \left(
			\hat n_{j} - \frac{\mathbb{I}}{2}
		\right) \\
		&= \sum_{j=1}^{L-1} \left[
			\frac{J_{xy}}{2} \left(
				\hat c_j^\dagger \hat c_{j+1} + \hat c_{j+1}^\dagger \hat c_j
			\right) + J_z \hat n_j \hat n_{j+1}
		\right] - \sum_{j=1}^L h_j \hat n_{j}
		+ \frac{hL}{2} + \frac{J_z (L-1)}{2}
	\end{aligned}
\]
where we defined:
\[
	h_j = \begin{cases}
		h + J_z/2 &\qq{if} j=1,L \\
		h + J_z &\qq{if} 1<j<L
	\end{cases}
\]
Apart from a constant energy shift, in the bulk ($1<j<L$) this is our spinless fermions hamiltonian in Eq.~\eqref{eq:spinless-hamiltonian-pbc} with the following maps:
\begin{equation}\label{eq:xxz-fermions-parameters-map}
	t = \frac{J_{xy}}{2}
	\qquad
	V = J_z
	\qquad
	\mu = h + J_z
\end{equation}

Now, add another site. This amounts for extra terms in the hamiltonian,
\[
	\hat H_\mathrm{XXZ} \to \hat H_\mathrm{XXZ} + \frac{J_{xy}}{2} \left( 
		\hat S_L^+ \hat S_{L+1}^- + \hat S_L^- \hat S_{L+1}^+
	\right) + J_z \hat S_L^z \hat S_{L+1}^z
\]
If we now aim to identify the open ends, which is, close the chain onto itself, the condition is
\[
	\hat S_{L+1}^\alpha = \hat S_1^\alpha
\]
The first-site Jordan-Wigner string is trivial, $\zeta_1 = 0$. The implementation of periodic boundary conditions requires in operator space then $\zeta_{L+1} \to \zeta_1 = 0$. Finally, it is trivial to see
\[
	(-1)^{\zeta_L} \hat c_L^\dagger = (-1)^{\#_F} \hat c_L^\dagger
	\qquad
	(-1)^{\zeta_L+1} \hat c_L = (-1)^{\#_F} \hat c_L
\]
with $\#_F$ the total number of fermions on the chain. Putting all together, we see that as long as $\#_F$ is \textbf{even} {\color{tabred}[Check: I have the suspect it should be odd, actually...]}, we can impose periodic boundary conditions,
\[
	\hat S_L^+ \hat S_1^- = \hat c_L^\dagger \hat c_1
	\qquad
	\hat S_L^- \hat S_1^+ = \hat c_1^\dagger \hat c_L
\]
Simpler reasoning holds for the $z$ terms. Putting all together, we see that the PBC-$\mathrm{XXZ}$ model is mapped onto a spinless fermionic system on a ring with even number of particles, with the rules of Eq.~\eqref{eq:xxz-fermions-parameters-map}. Since the PBC-$\mathrm{XXZ}$ admits for an exact Bethe-Ansatz solution, we can link the phase transitions of the two models.

\subsection*{Phase diagrams}

\begin{figure}
	\centering
	\begin{tikzpicture}[
		scale=2
	]
	\draw[color=black, -stealth]
		(-2,0) -- (2,0) node[anchor=west]
			{$J_z/J_{xy}$};
	\draw[color=black]
		(-1,-0.05) node [anchor=north] 
			{$-1$}
		-- (-1,0.05) node[anchor=south, color=tabblue] ()
			{HF}
		(0,-0.05) node [anchor=north] 
			{$0$}
		-- (0,0.05) node[anchor=south, color=tabblue]
			{XY}
		(1,-0.05) node [anchor=north] 
			{$1$} 
		-- (1,0.05) node[anchor=south, color=tabblue]
			{HAF};
		
		\node[anchor=south, color=tabblue]
			at (-1.5,0)
				{IF};
		\node[anchor=south, color=tabblue]
			at (1.5,0)
				{IAF};
\end{tikzpicture}
	\caption{Schematics for the $\mathrm{XXZ}$ model phase diagram. We are here considering a zero-field model, $h=0$ (which is mapped by the maps \eqref{eq:xxz-fermions-parameters-map} to a $\mu = J_z$ model). H indicates Heisenberg, I indicates Ising; F stands for Ferromagnet, AF for AntiFerromagnet; XY stands for pure $\mathrm{XY}$ model.}
	\label{xxz-phase-diagram}
\end{figure}

\subsection{Bosonization of the model}

\todo

\subsection{Phenomenology of spin-charge separation}

\todo

\subsection{Adding one effective interaction: the Extended Fermi-Hubbard model}

\todo