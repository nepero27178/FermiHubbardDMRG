\clearpage
\section{The Fermi-Hubbard model}

\begin{figure}
	\centering
	\subfloat[Lattice with null flux.]{
		\begin{tikzpicture}
	\foreach \i in {0,1,...,7}{
		\def\angle{45*\i}
		
		\draw[color=tabgreen]
			(\angle:2) -- ({\angle+45}:2);
			
		\node[anchor=center, color=tabgreen]
			at ({\angle+22.5}:2.2) {$t$};
		
		\filldraw[color=black]
			(\angle:2) circle (1pt);
	}
	
	\filldraw[color=black]
		(0:2) circle (1pt);
\end{tikzpicture}
		\label{subfig:lattice-ring}
	}
	\hfil
	\subfloat[Lattice with finite flux $\Phi$.]{
		\begin{tikzpicture}[
		decoration={
		markings,
		mark=at position 0.5 with {\arrow{stealth}}}
	]
	\foreach \i in {0,1,...,7}{
		\def\angle{45*\i}
		
		\draw[color=gray!50]
			(\angle:2) -- ({\angle+45}:2);
		
		\draw[color=tabgreen, , postaction={decorate}]
			(\angle:2.05) -- ({\angle+45}:2.05);
		
		\node[anchor=center, color=tabgreen]
			at ({\angle+22.5}:2.2) {$T$};
		
		\draw[color=tabblue, , postaction={decorate}]
			({\angle+45}:1.95) -- (\angle:1.95);
		
		\node[anchor=center, color=tabblue]
			at ({\angle+22.5}:1.45) {$T^*$};
		
		\filldraw[color=black]
			(\angle:2) circle (1pt);
	}
	
	\filldraw[color=black]
	(0:2) circle (1pt);
	
	\node[anchor=center] 
		at (0,0) {$\Phi$};
\end{tikzpicture}
		\label{subfig:lattice-ring-flux}
	}
	\caption{In Fig.~\ref{subfig:lattice-ring} a schematics of a $1D$ closed lattice is portrayed. The hopping amplitude $t$ is purely real, $t = \sgn(t) \abs{t}$. In Fig.~\ref{subfig:lattice-ring-flux} the same lattice is represented, but coupled to a finite threading flux $\Phi$ which can be absorbed via the pseudo-gauge transformation in Eq.~\todo. As a consequence, the hopping amplitude acquires a chirality which manifests in a non-null imaginary part, $T = t e^{i\Phi/L}$.}
	\label{fig:lattice-rings}
\end{figure}

In this project I limit myself to a polarized (spinless fermions) system. Extension to a spinful system is possible and introduces some refinements in the general bosonization scheme, the most notable being the famous spin-charge separation. Let me take it easy: consider the following interacting hamiltonian:
\begin{equation}\label{eq:spinless-hamiltonian-pbc}
	\hat H \equiv -t \sum_{\ev{j,k}} \left[
		\hat c_j^\dagger \hat c_k + \hat c_k^\dagger \hat c_j 
	\right] + V \sum_{\ev{j,k}} \hat n_j \hat n_k - \mu \sum_{j=1}^L \hat n_j
\end{equation}
defined on a closed $1D$ lattice ring, as in Fig.~\ref{subfig:lattice-ring}.
This is a simple nearest-neighbors (NN) interacting lattice hamiltonian with NN interaction $V$, chemical potential $\mu$ and hopping amplitude $t$,
\[
	t, V, \mu \in \mathbb{R}
\]

I will also be considering a magnetic flux $\Phi$ threading the ring and coupling to the charge degree of freedom. On a ring this pinned flux acts as a tangential vector potential, which is, a momentum offset; thus the correct way to absorb into our lattice framework this interaction is via the pseudo-gauge transformation
\[
	\hat c_j \to e^{-ij \phi} \hat c_j
	\qquad\qquad
	\hat c_j^\dagger \to e^{ij \phi} \hat c_j^\dagger
	\qquad\qquad
	\phi \equiv \frac{\Phi}{L}
\]
Incorporate the latter in the above hamiltonian: the hopping amplitude becomes complex (which is, chiral) $t \to T \equiv t e^{i\phi}$, with $t, \phi \in \mathbb{R}$. We have, as in Fig.~\ref{subfig:lattice-ring-flux}
\begin{equation}\label{eq:spinless-hamiltonian-tbc}
	\hat H \equiv -t \sum_{j=1}^L \left[ 
		e^{-i\phi} \hat c_j^\dagger \hat c_{j+1} + e^{i\phi} \hat c_{j+1}^\dagger \hat c_j 
	\right] + V \sum_{j=1}^L \hat n_j \hat n_{j+1} - \mu \sum_{j=1}^L \hat n_j
\end{equation}
where a $\mod L$ operation is intended: $L+1 \leftrightarrow 1$. I want to indagate its ground-state properties. The relevant parameters will be the reduced interaction $V/t$ and chemical potential $\mu/t$.
\[
	\begin{aligned}
		V/t &> 0 \qquad
		&&\text{Repulsive interaction} \\
		V/t &< 0 \qquad
		&&\text{Attractive interaction} \\
	\end{aligned}
\]
Intuitively, if the interaction becomes dominant with respect to the hopping dynamics, the combined effect of attraction/repulsion and Pauli exclusion principle should lead to two different forms of localization. On one hand, if the interaction is strong and attractive, the ground state should be uniformly filled, because fermions save energy both by closed packing and by increasing density due to the negative chemical potential contribution ($\mu > 0$). On the other hand, a strong repulsive interaction could lead to an half-filled chain, which sacrifices the chemical potential energy gain lost by reducing the particle number by not paying the energy cost of having nearby fermions. In both cases hopping is suppressed, thus fermions are localized; now it is a matter of seeing if these states are actually realized.

\subsection{Jordan-Wigner mapping of the Heisenberg XXZ model}

\begin{figure}
	\centering
	\begin{tikzpicture}
	
	% XXZ model
	\draw[color=gray!50]
		\foreach \i in {1,2,...,9}{
			(\i,0) --++ (1,0)
		};
	\fill[color=black]
		\foreach \i in {1,2,5,7,8,10}{
			(\i,0) circle (1pt) node[anchor=center]
				{$\uparrow$}
		}
		\foreach \i in {3,4,6,9}{
			(\i,0) circle (1pt) node[anchor=center]
			{$\downarrow$}
		};
	
	\draw[
		color=tabgreen,
		decorate,
		decoration={
			coil,
			aspect=0,
			amplitude=1pt,
			segment length=4pt
		}
	]
		(7,0.15) --++ (1,0) node[anchor=south, midway]
			{$+J_z$};
		
	\draw[
		color=tabgreen,
		decorate,
		decoration={
			coil,
			aspect=0,
			amplitude=1pt,
			segment length=4pt
		}
	]
		(8, -0.15) --++ (1,0) node[anchor=north, midway]
			{$-J_z$};
			
	\draw[
		color=tabblue,	
	]
		 (5,0.15) edge [-stealth, bend right=45] (4,0.15)
		 (4,-0.15) edge [-stealth, bend right=45] (5,-0.15);
		
	\node[color=tabblue, anchor=south]
		at (4.5, 0.3) {$J_{xy}/2$};
	\node[color=tabblue, anchor=north]
		at (4.5, -0.3) {$J_{xy}/2$};

	% Spinless FH model
	\draw[color=gray!50]
		\foreach \i in {1,2,...,9}{
			(\i,-2) --++ (1,0)
		};
	\filldraw[color=black, fill=black]
		\foreach \i in {1,2,5,7,8,10}{
			(\i,-2) circle (1.5pt) 
		};
	\filldraw[color=black, fill=white]
		\foreach \i in {3,4,6,9}{
			(\i,-2) circle (1.5pt)
		};
	
	\draw[
		color=tabgreen,
		decorate,
		decoration={
			coil,
			aspect=0,
			amplitude=0.5pt,
			segment length=6pt
		}
	]
		(7,-1.95) --++ (1,0) node[anchor=south, midway]
			{$+V$};
	
	\draw[
		color=tabgreen,
		decorate,
		decoration={
			coil,
			aspect=0,
			amplitude=0.5pt,
			segment length=6pt
		}
	]
		(8, -1.95) --++ (1,0) node[anchor=south, midway]
			{$-V$};
	
	\draw[
		color=tabblue,	
	]
		(5,-1.95) edge [-stealth, bend right=30] (4,-1.95)
		(4,-2.05) edge [-stealth, bend right=30] (5,-2.05);
	
	\node[color=tabblue, anchor=south]
		at (4.5, -1.85) {$-t$};
	\node[color=tabblue, anchor=north]
		at (4.5, -2.15) {$-t$};
	
	\node[align=left] 
		at (12, -1) {
			$\displaystyle
			\begin{aligned}
				t &= - J_{xy}/2 \\ 
				V &= J_z \\
				\mu &= h + J_z
			\end{aligned}$
			};
\end{tikzpicture}
	\caption{Schematics of the Jordan-Wigner mapping. The above chain represents the $\mathrm{XXZ}$ model, while the chain below represents the spinless Fermi-Hubbard model. Hollow circles represent holes, filled circles represents on-site particles. In both chain the two competing processes are represented: the NN interaction and the swapping interaction.}
	\label{fig:jordan-wigner-mapping}
\end{figure}

The model presented above can be obtained rather easily through a Jordan-Wigner of the Heisenberg XXZ model in transverse field,
\begin{equation}\label{eq:xxz-hamiltonian}
	\hat H_\mathrm{XXZ} \equiv \sum_{\ev{j,k}} \left[
		J_{xy} \left( 
			\hat S_j^x \hat S_k^x + \hat S_k^y \hat S_j^y
		\right) + J_z \hat S_j^z \hat S_k^z
	\right] - h \sum_{j=1}^L \hat S_j^z 
\end{equation}
The Jordan-Wigner mapping, only feasible in one dimension due to sites ordering, is given by:
\[
	\hat S_j^+ \to \hat c_j^\dagger e^{i\pi \sum_{k < j} \hat c_k^\dagger \hat c_k}
	\qquad
	\hat S_j^- \to \hat c_j e^{- i\pi \sum_{k < j} \hat c_k^\dagger \hat c_k}
	\qquad
	\hat S_j^z \to \hat n_j - \frac{\mathbb{I}}{2}
\]
It can be shown rather easily that, if the $c$-operators are canonically fermionic (which is, $\lbrace c_j, c_k^\dagger \rbrace = \delta_{jk}$) then the $3$D $\mathfrak{su}(2)$ spin algebra is preserved by this mapping. Notice the appearance of the Jordan parity operator up to site $j-1$, $\hat{\mathrm{P}}_{j-1}$, with
\[
	\hat{\mathrm{P}}_\ell = e^{i\pi \sum_{k \le \ell} \hat c_k^\dagger \hat c_k}
	\quad\to\quad
	(-1)^{\zeta_\ell}
	\qquad
	\zeta_\ell \equiv \sum_{k \le \ell} \hat n_k
\]
Essentially, the above string counts the fermions \textit{before} the site in question and gives back a factor $+1$ for even number, $-1$ for odd number. This works for open-ends chains, where the concept of \textit{before} is actually well-defined. 

Using basic algebra, it is straightforward to see:
\[
	\begin{aligned}
		S_j^+ \hat S_{j+1}^- &\to \left(
			\hat{\mathrm{P}}_{j-1} \hat c_j^\dagger
		\right) \left(
			\hat{\mathrm{P}}_j \hat c_{j+1}
		\right) = \hat c_j^\dagger \hat c_{j+1} \\
		S_j^- \hat S_{j+1}^+ &\to \left(
			\hat{\mathrm{P}}_{j-1} \hat c_j
		\right) \left(
			\hat{\mathrm{P}}_j \hat c_{j+1}^\dagger
		\right) = \hat c_{j+1}^\dagger \hat c_j \\
	\end{aligned}
\]
Indeed, the product operator $c_j^\dagger \hat{\mathrm{P}}_j$ vanishes if the site $j$ is occupied. Then site $j$ does not contribute to parity, giving $\mathrm{P}_{j-1}=\mathrm{P}_j$. Identical reasoning holds for the line below, completed by an anticommutation of Fermi operators. Let me take a $1\mathrm{D}$ spin chain with open boundary conditions (OBC). The transformation gives
\[
	\begin{aligned}
		\hat H_\mathrm{XXZ} &\equiv \sum_{j=1}^{L-1} \left[
			\frac{J_{xy}}{2} \left( 
				\hat S_j^+ \hat S_{j+1}^- + \hat S_j^- \hat S_{j+1}^+
			\right) + J_z \hat S_j^z \hat S_{j+1}^z
		\right] - h \sum_{j=1}^L \hat S_j^z \\
		&= \sum_{j=1}^{L-1} \left[
			\frac{J_{xy}}{2} \left(
				\hat c_j^\dagger \hat c_{j+1} + \hat c_{j+1}^\dagger \hat c_j
			\right) + J_z \left(
				\hat n_j - \frac{\mathbb{I}}{2}
			\right) \left(
				\hat n_{j+1} - \frac{\mathbb{I}}{2}
			\right)
		\right] - h \sum_{j=1}^L \left(
			\hat n_{j} - \frac{\mathbb{I}}{2}
		\right) \\
		&= \sum_{j=1}^{L-1} \left[
			\frac{J_{xy}}{2} \left(
				\hat c_j^\dagger \hat c_{j+1} + \hat c_{j+1}^\dagger \hat c_j
			\right) + J_z \hat n_j \hat n_{j+1}
		\right] - \sum_{j=1}^L h_j \hat n_{j}
		+ \frac{hL}{2} + \frac{J_z (L-1)}{2}
	\end{aligned}
\]
where I defined:
\[
	h_j = \begin{cases}
		h + J_z/2 &\qq{if} j=1,L \\
		h + J_z &\qq{if} 1<j<L
	\end{cases}
\]
Notice that for OBC the field term at the ends of the chain misses half the $J_z$ correction to the field, because of the missing interaction link.

Now I close the chain. This amounts to add a new interaction term,
\[
	\hat H_\mathrm{XXZ} \to \hat H_\mathrm{XXZ} + \frac{J_{xy}}{2} \left( 
		\hat S_L^+ \hat S_1^- + \hat S_L^- \hat S_1^+
	\right) + J_z \hat S_L^z \hat S_1^z
\]
Jordan-Wigner mapping requires a little more care here. Since
\[
	\begin{aligned}
		\hat S_L^+ \hat S_1^- &= \left(
			\hat{\mathrm{P}}_{L-1} \hat c_L^\dagger
		\right) \hat c_1 = \left(
			\hat{\mathrm{P}}_L \hat c_L^\dagger
		\right) \hat c_1 = (-1)^{\#_F -1} c_L^\dagger \hat c_1 \\
		\hat S_L^- \hat S_1^+ &= \left(
			\hat{\mathrm{P}}_{L-1} \hat c_L
		\right) \hat c_1^\dagger = \left(
			-\hat{\mathrm{P}}_L \hat c_L
		\right) \hat c_1^\dagger = -(-1)^{\#_F+1} c_L \hat c_1^\dagger = (-1)^{\#_F+1} \hat c_1^\dagger c_L
	\end{aligned}
\]
The first line holds because the $L$-th site must be empty, thus leaving unchanged the parity operator when extending $\mathrm{P}_{L-1} \to \mathrm{P}_L$, and simultaneously to evaluate the total chain parity after having applied $\hat c_1$ gives non-zero result only if the first site is occupied. Then, calling $\#_F$ the total number of fermions on the chain, the final parity is $\#-1$ which accounts for the first line sign prefactor. An analogous argument holds as well for the second line. We conclude that:
\begin{itemize}
	\item if the total number of fermions on the chain is \textbf{odd}, $\#_F = 2n+1$ for $n \in \mathbb{N}$, we can add the new interaction term to the hamiltonian with identical form,
	\begin{equation}\label{eq:xxz-hamiltonian-intermediate}
		\hat H_\mathrm{XXZ} = \sum_{j=1}^L \left[
			\frac{J_{xy}}{2} \left(
				\hat c_j^\dagger \hat c_{j+1} + \hat c_{j+1}^\dagger \hat c_j
			\right) + J_z \hat n_j \hat n_{j+1}
		\right] - (h+J_z) \sum_{j=1}^L \hat n_{j}
		\qquad
		\left(
			\hat c_{L+1} \equiv \hat c_1
		\right)
	\end{equation}
	I neglected an irrelevant energy shift. The PBC-$\mathrm{XXZ}$ model is mapped onto a spinless fermion system with odd number of fermions.
	\item if the total number of fermions on the chain is \textbf{even}, $\#_F = 2n+1$ for $n \in \mathbb{N}$, we can add the new interaction term to the hamiltonian closing the chain anti-periodically (or, equivalently, flipping the sign of the nearest-neighbor hopping term across the closing link),
	\[
		\hat H_\mathrm{XXZ} = \sum_{j=1}^L \left[
		\frac{J_{xy}}{2} \left(
				\hat c_j^\dagger \hat c_{j+1} + \hat c_{j+1}^\dagger \hat c_j
			\right) + J_z \hat n_j \hat n_{j+1}
		\right] - (h+J_z) \sum_{j=1}^L \hat n_{j}
		\quad
		\left(
			\hat c_{L+1} \equiv - \hat c_1
		\right)
	\]
	The APBC-$\mathrm{XXZ}$ model is mapped onto a spinless fermion system with even number of fermions.
\end{itemize}
For the sake of simplicity, I will limit this project to the first situation with an odd number of fermions. Since the PBC-$\mathrm{XXZ}$ admits for an exact Bethe-Ansatz solution, we can link the phase transitions of the two models.

Apart from a constant energy shift, the hamiltonian \eqref{eq:xxz-hamiltonian-intermediate} is \textit{de-facto} identical to the spinless model of Eq.~\eqref{eq:spinless-hamiltonian-pbc} via the mapping
\begin{equation}\label{eq:xxz-fermions-parameters-map}
	t = \frac{J_{xy}}{2}
	\qquad
	V = J_z
	\qquad
	\mu = h + J_z
\end{equation}
A schematics of this mapping\footnote{
	Note here that the \textit{exact} mapping between the two models up to this point would have been $t = - J_{xy}/2$. This sign flip is actually irrelevant for what concerns the goodness of the map: the zero-field PBC-$\mathrm{XXZ}$ phases are inverted under the canonical transformation given from sign-flipping the $xy$ spin components while leaving unchanged the $z$ one (see below). The same symmetry is implemented canonically for fermions by the fermionic-canonical transformation
	\[
	\hat c_i \to (-1)^i \hat c_i
	\]
	which is essentially a $\pi$ momentum shift. This transformation flips the hopping term sign leaving the others unchanged. Finally, this same sign problem would not have arisen if I would have chosen a slightly different but physically equivalent Jordan-Wigner mapping, connecting up-spins to holes and down-spins to particles.
} is given in Fig.~\ref{fig:jordan-wigner-mapping}.

\subsection*{Phase diagrams}

The phase diagram of the $\mathrm{XXZ}$ model is readily obtained by the means of exact methods like Bethe Ansatz. Of course, the dominant parameter is the ratio $J_z/J_{xy}$ which measures the dominant contribution to energy given by spin-spin NN interaction ($z$ term) and spin diffusion ($xy$ term). Whenever $\abs{J_z} = \abs{J_{xy}}$, the model is of the Heisenberg class.

\begin{figure}
	\centering
	\begin{tikzpicture}[
		scale=2
	]
	\draw[color=black, -stealth]
		(-2,0) -- (2,0) node[anchor=west]
			{$J_z/J_{xy}$};
	\draw[color=black]
		(-1,-0.05) node [anchor=north] 
			{$-1$}
		-- (-1,0.05) node[anchor=south, color=tabblue] ()
			{HF}
		(0,-0.05) node [anchor=north] 
			{$0$}
		-- (0,0.05) node[anchor=south, color=tabblue]
			{XY}
		(1,-0.05) node [anchor=north] 
			{$1$} 
		-- (1,0.05) node[anchor=south, color=tabblue]
			{HAF};
		
		\node[anchor=south, color=tabblue]
			at (-1.5,0)
				{IF};
		\node[anchor=south, color=tabblue]
			at (1.5,0)
				{IAF};
\end{tikzpicture}
	\caption{Schematics for the $\mathrm{XXZ}$ model phase diagram. We are here considering a zero-field model, $h=0$ (which is mapped by the maps \eqref{eq:xxz-fermions-parameters-map} to a $\mu = J_z$ model). H indicates Heisenberg, I indicates Ising; F stands for Ferromagnet, AF for Anti-Ferromagnet; XY stands for pure $\mathrm{XY}$ model. The color of each label recalls the dominant interaction of Fig.~\ref{fig:jordan-wigner-mapping}.}
	\label{fig:xxz-phase-diagram}
\end{figure}

We can safely assume $J_{xy}>0$ and only consider the relative sign of $J_z$. This is easily seen: if we map
\[
	J_{xy} \to - J_{xy}
	\qquad
	J_z \to J_z
\]
which is, exchange the hopping sign, this is equivalent to flipping the sign of the $xy$ spin component while leaving the $z$ component unchanged. Such a map is canonical (does not impact the spin algebra). The phase diagram for negative-sign hopping is anti-symmetric with respect to this inversion. The same holds for finite field. The basic, field free phase diagram is represented schematically in Fig.~\ref{fig:xxz-phase-diagram}.
\begin{enumerate}
	\item The system tends to an Ising Ferromagnet for $J_z / J_{xy} < -1$, with dominant behavior the complete alignment of spins (which maps onto the spinless Fermi-Hubbard model as a completely filled chain). 
	\item Moving across the first Heisenberg boundary, $J_z = - J_{xy}$, the dominant behavior for energy lowering is spin diffusion up to a perfect local fields free situation of $J_z = 0$. This phase maps onto the spinless Fermi-Hubbard model as a superconducting phase.
	\item Crossing the second Heisenberg boundary, $J_z = J_{xy}$, the system tends to an Ising Anti-Ferromagnet, dominated by the Néel state configuration. The latter maps onto an half-filling and Mott-localized fermionic chain.
\end{enumerate}
For a fixed number system, this is the expected phase dynamics.

\begin{figure}
	\centering
	\begin{tikzpicture}
	
	\begin{axis}[
			axis lines=center,
			axis on top,
			height=0.5\textwidth,
			width=0.75\textwidth,
			xlabel={$\Delta$},		ylabel={$h/J$},
			xlabel style={right}, 	ylabel style={above},
			xtick={-1,0,1},			ytick={-1,0,1},
			xticklabel style={yshift=-0.1cm},
			yticklabel style={xshift=-0.2cm},
			xmin=-1.5, 				ymin=-2,
			xmax=2.5, 				ymax=2
		]
		
		\addplot[name path=HAFUpCurved, domain=1:1.5, dashed] 
			{2*(x-1)^4};
		\addplot[name path=HAFUpStraight, domain=1.5:2.5, dashed] 
			{x-1.375};
		\addplot[name path=HAFDownCurved, domain=1:1.5, dashed] 
			{-2*(x-1)^4};
		\addplot[name path=HAFDownStraight, domain=1.5:2.5, dashed] 
			{-x+1.375};
		\addplot[name path=HFUp1, domain=-1:1.5, dashed] 
			{x+1};
		\addplot[name path=HFUp2, domain=1.5:2.5, dashed] 
			{x+1};
		\addplot[name path=HFDown1, domain=-1:1.5, dashed] 
			{-x-1};
		\addplot[name path=HFDown2, domain=1.5:2.5, dashed] 
			{-x-1};
			
		\addplot [tabblue!25] 
			fill between [of = HAFUpCurved and HFUp1];
		\addplot [tabblue!25] 
			fill between [of = HAFUpStraight and HFUp2];
		\addplot [tabblue!25] 
			fill between [of = HAFDownCurved and HFDown1];
		\addplot [tabblue!25] 
			fill between [of = HAFDownStraight and HFDown2];
			
		\addplot[color=tabgreen!25]
			fill between [of = HAFUpCurved and HAFDownCurved];
		\addplot[color=tabgreen!25]
			fill between [of = HAFUpStraight and HAFDownStraight];
			
		\path[name path=SupportUp1] (-1.5,3) -- (-1,3);
		\path[name path=SupportUp2] (-1,3) -- (2.5,3);
		\path[name path=SupportDown1] (-1.5,-3) -- (-1,-3);
		\path[name path=SupportDown2] (-1,-3) -- (2.5,-3);
		
		\addplot[color=tabred!25]
			fill between [of = SupportUp1 and SupportDown1];
		\addplot[color=tabred!25]
			fill between [of = SupportUp1 and HFUp1];
		\addplot[color=tabred!25]
			fill between [of = SupportDown1 and HFDown1];
		
		\node[anchor=south east, color=tabblue]
			at (axis cs:0,0) {XY};
		\node[anchor=south east, color=tabred]
			at (axis cs:-1,0) {HF};
		\node[anchor=south west, color=tabgreen]
			at (axis cs:1.75,0) {HAF};
		
		\fill
			(axis cs:-1,0) circle (1pt)
			(axis cs:1,0) circle (1pt);
	
	\end{axis}
\end{tikzpicture}
	\caption{Schematics of the phase boundaries for the $\mathrm{XXZ}$ model as extracted by \citeauthor{rakov2016symmetries} \cite{rakov2016symmetries}. Notice the two zero-field phase-transition points $(-1,0)$ and $(1,0)$.}
	\label{fig:xxz-field-phase-diagram}
\end{figure}

Now, define $J$ and $\Delta$ as in \cite{rakov2016symmetries}, which is
\[
	J_{xy} \equiv J
	\qquad\qquad
	J_z \equiv J\Delta
\]
This gives the mapping:
\begin{equation}\label{eq:xxz-fermions-parameters-map-2}
	t = \frac{J}{2}
	\qquad
	\frac{V}{t} = 2\Delta
	\qquad
	\frac{\mu}{t} = 2 \left(
		\frac{h}{J} + \Delta
	\right)
\end{equation}
Within this parametrization, the analytical phase diagram is the one depicted in Fig.~\ref{fig:xxz-field-phase-diagram} \cite{rakov2016symmetries}.
Now, I use $\mu/t > 0$. This maps on the $\mathrm{XXZ}$ model as $h/J > -\Delta$, a condition depicted in Fig.~\ref{fig:jordan-wigner-mapping}. Thus, running simulations in a regime $\mu/t > 0$ and $V/t \in \mathbb{R}$, what one expects is to encounter all three phases. Now, I need to reconnect this expectation in a bosonization framework.

\begin{figure}
	\centering
	\begin{tikzpicture}
	
	\begin{axis}[
			axis lines=center,
			axis on top,
			height=0.5\textwidth,
			width=0.75\textwidth,
			xlabel={$\Delta$},		ylabel={$h/J$},
			xlabel style={right}, 	ylabel style={above},
			xtick={1},				ytick={1},
			xticklabel style={below},
			yticklabel style={below right, xshift=0.1cm},
			xmin=-1.5, 				ymin=-2,
			xmax=2.5, 				ymax=2
		]
		\addplot[name path=HAFUpCurved, domain=1:1.5, dashed] 
			{2*(x-1)^4};
		\addplot[name path=HAFUpStraight, domain=1.5:2.5, dashed] 
			{x-1.375};
		\addplot[name path=HAFDownCurved, domain=1:1.5, dashed] 
			{-2*(x-1)^4};
		\addplot[name path=HAFDownStraight, domain=1.5:2.5, dashed] 
			{-x+1.375};
		\addplot[name path=HFUp1, domain=-1:1.5, dashed] 
			{x+1};
		\addplot[name path=HFUp2, domain=1.5:2.5, dashed] 
			{x+1};
		\addplot[name path=HFDown1, domain=-1:1.5, dashed] 
			{-x-1};
		\addplot[name path=HFDown2, domain=1.5:2.5, dashed] 
			{-x-1};
		\addplot[name path=Border, domain=-1.5:2.5]
			{-x};
		\path[name path=SupportDown] 
			(-1.5,-3) -- (2.5,-3);
		\addplot[pattern=north west lines, pattern color=gray!50]
			fill between [of = SupportDown and Border];

		\node[anchor=south west, color=tabblue]
			at (axis cs:0,0) {XY};
		\node[anchor=south east, color=tabred]
			at (axis cs:0,1) {HF};
		\node[anchor=south west, color=tabgreen]
			at (axis cs:1.75,0) {HAF};
		\node[anchor=north east] 
			at (axis cs:2.5,2) {$\mu/t > 0$};
		\node[anchor=south west] 
			at (axis cs:-1.5,-2) {$\mu/t < 0$};

	\end{axis}
\end{tikzpicture}
	\caption{The same phase diagram as in Fig.~\ref{fig:xxz-field-phase-diagram}, completed with the requirement $h/J < \Delta$. The shaded area is excluded from the mapping. Note that a positive chemical potential excludes the Antiferromagnetic phase.}
	\label{fig:jordan-wigner-phase-diagram}
\end{figure}

\subsection{Non-interacting ground-state}

In order to further explore the interacting system, we need to know its zero-field non-interacting ground-state. This amounts to setting $V=\mu=0$. The hamiltonian is well known,
\[
	\hat H_0 = -t \sum_{j=1}^{L} \left(
		\hat c_j^\dagger \hat c_{j+1} + \hat c_{j+1}^\dagger \hat c_j
	\right)
\]
and is easily solved by a simple Fourier transformation,
\[
	\hat c_j \equiv \sum_k e^{ikj} \hat c_k
\]
(I am here using adimensional momenta $k$, since space has throughout been considered as a simple integer index) which leads to
\begin{equation}\label{eq:XY-hamiltonian}
	\hat H_0 = -t \sum_k \left[
		e^{-ik} \hat c_k^\dagger \hat c_k + e^{ik} \hat c_k^\dagger \hat c_k
	\right] = \sum_k \epsilon_k \hat c_k^\dagger \hat c_k
	\quad\text{for}\quad
	\epsilon_k = -2t \cos k
\end{equation}
The band $\epsilon_k$ is represented in Fig.~\ref{fig:XY-band}. At half-filling, this simple sinusoidal band presents a null Fermi energy at Fermi wavevector $k_F = \pm \pi/2$. Linearization of the band for subsequent bosonization is as well represented in Fig.~\ref{fig:XY-band}.

\begin{figure}
	\centering
	\begin{tikzpicture}
	\begin{axis}[
			axis x line=center,
			axis y line=left,
			xlabel={$k$},
			ylabel={Energy},
			xlabel style={below},
			ylabel style={above},
			xtick={-pi,0,pi},
			ytick={-1,0,1},
			extra x ticks={-pi/2,pi/2},
			extra x tick labels=\empty,
			xticklabels={$-\pi$,$0$,$\pi$},
			yticklabels={$-2t$, $0$, $2t$},
			xmin=-4.0, xmax=4.0,
			ymin=-2.0, ymax=2.0
		]
		\addplot[domain=-pi:pi, smooth, color=tabblue]
			{-cos(deg(x))} node[below] {$\epsilon_k$};
		
		\addplot[domain=0.2:-pi, color=tabgreen]
			{-1.57-x} node[right] {$\epsilon_k^{(\mathrm L)}$};
		
		\addplot[domain=-0.2:pi, color=tabgreen]
			{-1.57+x} node[left] {$\epsilon_k^{(\mathrm R)}$};
	\end{axis}
\end{tikzpicture}
	\caption{Excitations band and relative linearizations for the non-interacting model ($V=0$). At half-filling the ground-state has null Fermi energy, the negative part of the band is filled and the spectrum is gapless.}
	\label{fig:XY-band}
\end{figure}

\subsection{Bosonization of the free model}

I now \textit{bosonize} the spinless Fermi-Hubbard model. Recall the processes of Fig.~\ref{fig:g-processes}: for a spinless system the only contributions to a bosonized hamiltonian is from the $g_2$ and $g_4$ processes. The non-interacting hamiltonian is very simple, and it already known how to reduce it to the form of Eq.~\eqref{eq:free-field-hamiltonian}: all is needed is to take a coherent continuum limit for the fermionic operators,
\[
	\hat c_j \to \hat \psi(x)
\]
Let me reintroduce a lattice spacing $a$ for the sake of dimensional correctness. The bosonization of an hamiltonian of the kind of Eq.~\eqref{eq:XY-hamiltonian} has been treated at the very beginning of the present report. The only step to be done is the linearization depicted in Fig.~\ref{fig:XY-band},
\[
	-2t \cos (ka) \big|_{k = \pm \frac{\pi}{2a} + q} = 2t \cos \left(
		\pm \frac{\pi}{2} + qa
	\right) \simeq \pm (a J) q
\]
I take as granted the result of Eq.~\eqref{eq:free-field-hamiltonian}, recognizing the Fermi velocity simply as
\[
	v_F = \partial_k \epsilon_k \big|_{k = k_F} = a J
\]
(As always, I omit a term $\hbar=1$).
For the sake of completeness, here I also derive the Dirac expression for the hamiltonian density of these linear dispersion fermions (which, of course, are massless). First, split the hamiltonian of Eq.~\eqref{eq:XY-hamiltonian},
\[
	\hat H_0 \simeq \hat H_0^{(\mathrm{R})} + \hat H_0^{(\mathrm{L})}
	\quad\text{with}\quad
	\hat H_0^{(s)} \equiv \sgn(s) v_F \sum_q q \left[ \hat c_k^{(s)} \right]^\dagger \hat c_k^{(s)}
\]
I take a continuum limit: with fixed $L$, I employ $a \to 0^+$. Then, transforming back to space,
\[
	\hat c_k^{(s)} = \frac{1}{\sqrt{L}} \int_0^L dx \, e^{-ikx} \hat \psi_s(x) 
\]
Recall the shifted fields $\hat \Psi_s(x)$ of Eq.~\eqref{eq:shifted-fermionic-fields-definition}. Being $q$ defined with respect to the Fermi point $k = \sgn(s) k_F$, such that $q = k - \sgn(s) k_F$,
\[
	\hat c_k^{(s)} = \frac{1}{\sqrt{L}} \int_0^L dx \, e^{-iqx} \hat \Psi_s(x)
\]
Then
\[
	H_0^{(s)} = \sgn(s) \frac{v_F}{L} \sum_q q \int_0^L dx \, e^{iqx} \hat \Psi_s^\dagger(x) \int_0^L dy \, e^{-iqy} \hat \Psi_s(y)
\]
Since
\[
	\partial_y \left[
		e^{-iqy} \hat \Psi_s(y)
	\right] = -iq e^{-iqy} \hat \Psi_s(y) + e^{-iqy} \partial_x \hat \Psi_s(y)
\]
and integrating the left-hand side one gets zero, being the chain closed with PBC,
\[
	H_0^{(s)} = -i\sgn(s) \frac{v_F}{L} \int_0^L dx \, \hat \Psi_s^\dagger(x) \int_0^L dy \, \partial_x \hat \Psi_s(y) \sum_q e^{iq(x-y)}
\]
Using the continuum limit
\[
	\frac{1}{L} \sum_q
	\quad\to\quad
	\frac{1}{2\pi} \int_\mathbb{R} dq 
\]
the $s$-side hamiltonian finally reduces to
\[
	H_0^{(s)} = -i\sgn(s) \frac{v_F}{2\pi} \int_0^L dx \, \hat \Psi_s^\dagger(x) \partial_x \hat \Psi_s(x)
\]
Then the massless chiral Dirac hamiltonian is recovered:
\begin{equation}\label{eq:massless-Dirac-hamiltonian}
	\hat H_0 = \frac{v_F}{2\pi i} \int_0^L dx \, \left[
		\hat \Psi_\mathrm{R}^\dagger(x) \partial_x \hat \Psi_\mathrm{R}(x) - \hat \Psi_\mathrm{L}^\dagger(x) \partial_x \hat \Psi_\mathrm{L}(x)
	\right]
\end{equation}
Now, all is left to do is to insert interactions in the bosonized hamiltonian and link the model parameters to the Luttinger parameters $u$ and $K$.

\subsection{Bosonization of nearest-neighbors interaction}

The interaction term is of the density-density class,
\[
	\hat V = V \sum_{j=1}^L \hat n_j \hat n_{j+1} = a^2 J\Delta \sum_{j=1}^L \frac{\hat n_j}{a} \frac{\hat n_{j+1}}{a}
\]
We identify $\hat n_j/a \to \hat \rho(x=ja)$ as the continuum density, thus giving (splitting in left/right contributions)
\begin{equation}\label{eq:interaction-hamiltonian-lattice-density}
	\hat V = a^2 J\Delta \sum_{j=1}^L \hat \rho(ja) \hat \rho\left(
		\vphantom{A^A} (j+1)a
	\right)
\end{equation}
The detailed derivation can be found in \cite[Sec.~6.1.2]{giamarchi2003quantum}. The starting point is the expansion of $\hat \rho$, carried out in terms of physical fields $\hat \psi_s$ or shifted fields $\hat \Psi_s$ as defined in Eq.~\eqref{eq:shifted-fermionic-fields-definition},
\begin{equation}\label{eq:density-operator-shifted-fermionic-fields}
	\begin{aligned}
		\hat \rho(x) &= - \frac{1}{\pi} \nabla \hat \phi(x) + \left[
			\hat \psi_\mathrm{R}^\dagger(x) \hat \psi_\mathrm{L}(x) + \hat \psi_\mathrm{L}^\dagger(x) \hat \psi_\mathrm{R}(x) 
		\right] \\
		&= - \frac{1}{\pi} \nabla \hat \phi(x) + \left[
			e^{-2ik_F x}
			\hat \Psi_\mathrm{R}^\dagger(x) \hat \Psi_\mathrm{L}(x) + e^{2ik_F x} \hat \Psi_\mathrm{L}^\dagger(x) \hat \Psi_\mathrm{R}(x) 
		\right]
	\end{aligned}
\end{equation}
In thermodynamic limit, recalling Eq.~\eqref{eq:fermionic-fields-bosonic-fields-expression},
\[
	\hat \Psi_s (x) = \frac{\hat U_s}{\sqrt{2\pi\alpha}} \exp\left\{
		i \left(
			\sgn(s) \hat \phi(x) + \hat \theta(x)
		\right)
	\right\}
	\qquad\left( \alpha \to 0 \right)
\]
The analytical calculation gets a little cumbersome: I only highlight the essential passage. First, express analytically the density operator of Eqns.~\eqref{eq:density-operator-shifted-fermionic-fields} using the results
\[
	\hat \Psi_\mathrm{R}^\dagger(x) \hat \Psi_\mathrm{L}(x) = \frac{\hat U_\mathrm{R}^\dagger \hat U_\mathrm{L}}{2\pi\alpha} e^{-2i\hat \phi(x)}
	\qquad
	\hat \Psi_\mathrm{L}^\dagger(x) \hat \Psi_\mathrm{R}(x) = \frac{\hat U_\mathrm{L}^\dagger \hat U_\mathrm{R}}{2\pi\alpha} e^{2i\hat \phi(x)}
\]
which gives
\begin{equation}\label{eq:density-operator-bosonic-fields}
	\hat \rho(x) = - \frac{1}{\pi} \nabla \hat \phi(x) + \left[
		e^{-2ik_F x}
		\frac{\hat U_\mathrm{R}^\dagger \hat U_\mathrm{L}}{2\pi\alpha} e^{-2i\hat \phi(x)}
		+ e^{2ik_F x}
		\frac{\hat U_\mathrm{L}^\dagger \hat U_\mathrm{R}}{2\pi\alpha} e^{2i\hat \phi(x)} 
	\right]
\end{equation}
Insert now the above result in Eq.~\eqref{eq:interaction-hamiltonian-lattice-density} for both the densities appearing. The key property to be used here is that the shifted fields $\hat \Psi_s$ (and hence the bosonic fields) vary \textit{slowly} with respect to $k_F$. This means that, in the limit $a \to 0$ of null lattice spacing, one can approximate the spatial variation as only happening in the fast-oscillating exponents of the above equation. This assumption limits this derivation to slight fluctuations of the fields around equilibrium, which means, to perturbative results. Moreover, one identifies the cutoff $\alpha$ with $a$, sending both to zero. The two densities multiplied in Eq.~\eqref{eq:interaction-hamiltonian-lattice-density} then give rise to $3 \times 3 = 9$ terms, of which the most relevant are
\begin{itemize}
	\item the one coming from the $\phi$ field gradients multiplied,
	\[
		\frac{1}{\pi^2} \nabla \hat \phi(x) \nabla \hat \phi(x+a) \simeq \frac{1}{\pi^2} \left( \nabla \hat \phi(x) \right)^2
	\]
	Such approximation is possible because, as said, fields vary slowly and then $x+a$ can be taken to be $x$ as well when a field variable;
	\item the one coming from the multiplication of the second term of Eq.~\eqref{eq:density-operator-bosonic-fields} (evaluated at $x$) with the third one (evaluated at $x+a$)
	\[
		e^{-2ik_F x}
		\frac{\hat U_\mathrm{R}^\dagger \hat U_\mathrm{L}}{2\pi a} e^{-2i\hat \phi(x)} \times
		e^{2ik_F (x+a)}
		\frac{\hat U_\mathrm{L}^\dagger \hat U_\mathrm{R}}{2\pi a} e^{2i\hat \phi(x+a)} = \frac{1}{(2\pi a)^2} e^{2ik_F a} e^{2i \left[ \hat \phi(x+a) - \hat \phi(x) \right]}
	\]
	where I dropped the Klein factors and substituted $\alpha \to a$;
	\item the hermitian conjugate of the above term, which arises inverting the roles in the previous point.
\end{itemize}
The other terms either vanish when summed to their respective hermitian conjugate or are subdominant\footnote{
	A word of caution fits here. Another term actually appears in the expansion, namely a \textit{umklapp} term. As is explained by \citeauthor{giamarchi2003quantum} in \cite[Sec.~6.1.2]{giamarchi2003quantum}, for a spinless hamiltonian defined on a lattice this term is less relevant when compared to the others (but is far from irrelevancy for a complete physical description!).
}. Then
\[
	\hat \rho(x) \hat \rho (x+a) \simeq \frac{1}{\pi^2} \left( \nabla \hat \phi(x) \right)^2
	+ \frac{1}{(2\pi a)^2} e^{2ik_F a} e^{2i \left[ \hat \phi(x+a) - \hat \phi(x) \right]} + \frac{1}{(2\pi a)^2} e^{-2ik_F a} e^{-2i \left[ \hat \phi(x+a) - \hat \phi(x) \right]}
\]
Finally, approximating
\[
	\begin{aligned}
		e^{\pm 2i \left[ \hat \phi(x+a) - \hat \phi(x) \right]} &\simeq e^{\pm 2ia \nabla \hat \phi(x)} \\
		&\simeq 1 \pm 2ia \nabla \hat \phi(x) - 2a^2 \left( \nabla \hat \phi(x) \right)^2
	\end{aligned}
\]
The $1$ constant term can be discarded: it only generates a constant energy shift. The linear terms $ \pm 2ia \nabla \hat \phi(x)$ can be discarded as well, since they vanish when summed. Only the last survives:
\[
	\begin{aligned}
		\hat \rho(x) \hat \rho (x+a) &\simeq \frac{1}{\pi^2} \left( \nabla \hat \phi(x) \right)^2 
		- \frac{2a^2}{(2\pi a)^2} \left( \nabla \hat \phi(x) \right)^2 \left(
			e^{2ik_F a} + e^{-2ik_F a}
		\right) \\
		&= \frac{1}{\pi^2} \left( \nabla \hat \phi(x) \right)^2 \left(
			\vphantom{A^A}
			1 - \cos(2k_F a)
		\right) \\
		\text{(at half filling)} &= \frac{2}{\pi^2} \left( \nabla \hat \phi(x) \right)^2
	\end{aligned}
\]
It's done. Recalling the bosonic structure of the general bosonized spinless Luttinger hamiltoninan of Eq.~\eqref{eq:interacting-fields-hamiltonian}, we finally conclude that the bosonized interacting hamiltonian is given by
\begin{equation}\label{eq:XXZ-luttinger-hamiltonian}
	\frac{1}{2\pi} \int_0^L dx \, \left[ \frac{u}{K} \left( \nabla \hat \phi(x) \right)^2 + uK \left( \nabla \hat \theta(x) \right)^2 \right]
	\quad\text{with}\quad
	\begin{cases}
		uK &= aJ \\
		u/K &= aJ \left(
			1 + 4\Delta/\pi
		\right)
	\end{cases}
\end{equation}
This simple result is a consequence of the many approximations I made. It is far from exact and strictly perturbative. By the means of Bethe Ansatz, a much better result can be obtained. It must be said, by the time the author got to this point, the hopes for the simulated data to be anywhere near any of these results were narrow.